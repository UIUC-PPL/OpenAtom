\documentclass[paper=a4, fontsize=11pt]{article} % A4 paper and 11pt font size

\usepackage[T1]{fontenc} % Use 8-bit encoding that has 256 glyphs

\usepackage{geometry}
 \geometry{
 a4paper,
 total={170mm,257mm},
 left=20mm,
 top=20mm,
 }
 
\usepackage{physics}


\usepackage[english]{babel} % English language/hyphenation
\usepackage{amsmath,amsfonts,amsthm} % Math packages
\usepackage{graphicx}
\usepackage{parskip}
\graphicspath{ {images/} }


\numberwithin{equation}{section} % Number equations within sections (i.e. 1.1, 1.2, 2.1, 2.2 instead of 1, 2, 3, 4)
\numberwithin{figure}{section} % Number figures within sections (i.e. 1.1, 1.2, 2.1, 2.2 instead of 1, 2, 3, 4)
\numberwithin{table}{section} % Number tables within sections (i.e. 1.1, 1.2, 2.1, 2.2 instead of 1, 2, 3, 4)

\setlength\parindent{20pt} % Removes all indentation from paragraphs - comment this line for an assignment with lots of text
\setlength{\parskip}{11pt}
\newcommand{\p}{\partial}
\newcommand{\wt}{\widetilde}
\newcommand{\ol}{\overline}
\newcommand{\bh}{{\bf{h}}}
\newcommand{\bc}{{\bf{c}}}
\newcommand{\bu}{{\bf{u}}}
\newcommand{\bv}{{\bf{v}}}
\newcommand{\bm}{{\bf{m}}}
\newcommand{\bl}{{\bf{l}}}
\newcommand{\bk}{{\bf{k}}}
\newcommand{\bK}{{\bf{K}}}
\newcommand{\bs}{{\bf{s}}}
\newcommand{\bg}{{\bf{g}}}
\newcommand{\bt}{{\bf{t}}}
\newcommand{\bG}{{\bf{G}}}
\newcommand{\br}{{\bf{r}}}
\newcommand{\bR}{{\bf{R}}}
\newcommand{\hs}{{\hat{\bf{s}}}}
\newcommand{\rexp}{{\mathrm{exp}}}
\newcommand{\rS}{{\mathrm{S}}}
\newcommand{\rKE}{{\mathrm{KE}}}
\newcommand{\rEES}{{\mathrm{EES}}}
\newcommand{\rxc}{{\mathrm{xc}}}
\newcommand{\rconv}{{\mathrm{conv}}}
\newcommand{\rgr}{{\mathrm{group}}}
\newcommand{\rcore}{{\mathrm{core}}}
\newcommand{\rcut}{{\mathrm{cut}}}
\newcommand{\rtol}{{\mathrm{tol}}}
\newcommand{\rNN}{{\mathrm{NN}}}
\newcommand{\rself}{{\mathrm{self}}}
\newcommand{\re}{{\mathrm{e}}}
\newcommand{\rCo}{{\mathrm{Coul}}}
\newcommand{\rshort}{{\mathrm{short}}}
\newcommand{\rcc}{{\mathrm{c.c.}}}
\newcommand{\rlong}{{\mathrm{long}}}
\newcommand{\rerf}{{\mathrm{erf}}}
\newcommand{\rerfc}{{\mathrm{erfc}}}
\newcommand{\rP}{{\mathrm{PAW}}}
\newcommand{\rnS}{{\mathrm{nonS}}}
\newcommand{\rd}{{\mathrm{d}}}
\newcommand{\rKS}{{\mathrm{KS}}}
\newcommand{\rFFT}{{\mathrm{FFT}}}
\newcommand{\rH}{{\mathrm{H}}}
\newcommand{\rA}{{\mathrm{A}}}
\newcommand{\rB}{{\mathrm{B}}}
\newcommand{\rD}{{\mathrm{D}}}
\newcommand{\rcomp}{{\mathrm{comp}}}
\newcommand{\rdet}{{\mathrm{det}}}
\newcommand{\rsum}{{\mathrm{sum}}}
\newcommand{\rcomb}{{\mathrm{comb}}}
\newcommand{\rgrid}{{\mathrm{grid}}}
\newcommand{\rlo}{{\mathrm{loc}}}
\newcommand{\rl}{{\mathrm{log}}}
\newcommand{\rtot}{{\mathrm{tot}}}
\newcommand{\ibgR}{i\bg\cdot\bR}
\newcommand{\ibgr}{i\bg\cdot\br}
\newcommand{\ibgpR}{i\bg'\cdot\bR}
\newcommand{\ibgpr}{i\bg'\cdot\br}
\newcommand{\gc}{{\bg_c}}
\newcommand{\gcn}{{\bg_c^{(n)}}}
\newcommand{\gcpEES}{{\bg_c^{(\Psi,\rEES)}}}
\newcommand{\gcnEES}{{\bg_c^{(n,\rEES)}}}
\newcommand{\igcs}{2\pi i\gc\cdot\hat \bs}
\newcommand{\igpcs}{2\pi i\gc'\cdot\hat \bs}
\newcommand{\igcns}{2\pi i\gcn\cdot\hat \bs}
\newcommand{\igpcns}{2\pi i\gcn'\cdot\hat \bs}
\newcommand{\igcps}{2\pi i\gcpEES\cdot\hat \bs}
\newcommand{\igpcps}{2\pi i\gcpEES'\cdot\hat \bs}
\newcommand{\igcnEESs}{2\pi i\gcnEES\cdot\hat \bs}
\newcommand{\igpcnEESs}{2\pi i\gcnEES'\cdot\hat \bs}
\newcommand{\pabc}{\prod_{\alpha=a,b,c}}
\newcommand{\al}{{\alpha}}
\newcommand{\psigs}{{\overline \Psi_I^{(\rS)}(\bg)}}
\newcommand{\psigsc}{{\overline \Psi_I^{(\rS)*}(\bg)}}
\newcommand{\psigps}{{\overline \Psi_I^{(\rS)}(\bg')}}
\newcommand{\psigpsc}{{\overline \Psi_I^{(\rS)*}(\bg')}}
\newcommand{\psiIpgs}{{\overline \Psi_{I'}^{(\rS)}(\bg)}}
\newcommand{\psiIpgsc}{{\overline \Psi_{I'}^{(\rS)*}(\bg)}}
\newcommand{\psiIpgps}{{\overline \Psi_{I'}^{(\rS)}(\bg')}}
\newcommand{\psiIpgpsc}{{\overline \Psi_{I'}^{(\rS)*}(\bg')}}
\newcommand{\sigs}{{\overline \Psi_I(\bg)}}
\newcommand{\sigsc}{{\overline \Psi_I^{*}(\bg)}}
\newcommand{\sigps}{{\overline \Psi_I(\bg')}}
\newcommand{\sigpsc}{{\overline \Psi_I^{*}(\bg')}}
\newcommand{\siIpgs}{{\overline \Psi_{I'}(\bg)}}
\newcommand{\siIpgsc}{{\overline \Psi_{I'}^{*}(\bg)}}
\newcommand{\siIpgps}{{\overline \Psi_{I'}(\bg')}}
\newcommand{\siIpgpsc}{{\overline \Psi_{I'}^{*}(\bg')}}
\newcommand{\RJa}{{R_{J,\alpha}}}
\newcommand{\RJb}{{R_{J,\beta}}}
\newcommand{\NKS}{{N_{\mathrm{KS}}}}
\newcommand{\NFFT}{{N_{\mathrm{FFT}}}}
\newcommand{\NFFTn}{{N^{(n)}_{\mathrm{FFT}}}}
\newcommand{\NFFTp}{{N^{\mathrm{(\Psi})}_{\mathrm{FFT}}}}
\newcommand{\NFFTnEES}{{N^{(n,\rEES)}_{\mathrm{FFT}}}}
\newcommand{\NFFTpEES}{{N^{\mathrm{(\Psi,\rEES})}_{\mathrm{FFT}}}}
\newcommand{\NgKS}{{N^{\mathrm{(\Psi})}_\bg}}
\newcommand{\Ngp}{{N^{\mathrm{(\Psi})}_\bg}}
\newcommand{\Ngn}{{N^{(n)}_\bg}}
\newcommand{\Rc}{{R_{\mathrm{c}}}}
\newcommand{\Gc}{{G_{\mathrm{c}}}}
\newcommand{\hGc}{{\frac{G_{\mathrm{c}}}{2}}}
\newcommand{\Rpc}{{R_{\mathrm{pc}}}}
\newcommand{\Dng}{{D^{(n)}_p(\bg)}}
\newcommand{\Dpg}{{D^{(\Psi)}_p(\bg)}}
\newcommand{\Dngc}{{D^{(n)*}_p(\bg)}}
\newcommand{\Dpgc}{{D^{(\Psi)*}_p(\bg)}}
\newcommand{\Dngp}{{D^{(n)}_p(\bg')}}
\newcommand{\Dpgp}{{D^{(\Psi)}_p(\bg')}}
\newcommand{\Dngcp}{{D^{(n)*}_p(\bg')}}
\newcommand{\Dpgcp}{{D^{(\Psi)*}_p(\bg')}}
\newcommand{\Mn}{{M_p^{(3,n)}}}
\newcommand{\Mp}{{M_p^{(3,\Psi)}}}
\newcommand{\FFTn}{{\mathrm{FFT}^{(n,+)}}}
\newcommand{\FFTnEES}{{\mathrm{FFT}^{(n,+,\rEES)}}}
\newcommand{\FFTnm}{{\mathrm{FFT}^{(n,\pm)}}}
\newcommand{\FFTnmEES}{{\mathrm{FFT}^{(n,\pm,\rEES)}}}
\newcommand{\FFTp}{{\mathrm{FFT}^{(\Psi,+)}}}
\newcommand{\FFTpm}{{\mathrm{FFT}^{(\Psi,\pm)}}}
\newcommand{\FFTpEES}{{\mathrm{FFT}^{(\Psi,+,\rEES)}}}
\newcommand{\FFTpmEES}{{\mathrm{FFT}^{(\Psi,\pm,\rEES)}}}
\newcommand{\IFFTn}{{\mathrm{IFFT}^{(n,+)}}}
\newcommand{\IFFTnEES}{{\mathrm{IFFT}^{(n,+,\rEES)}}}
\newcommand{\IFFTp}{{\mathrm{IFFT}^{(\Psi,+)}}}
\newcommand{\IFFTpEES}{{\mathrm{IFFT}^{(\Psi,+,\rEES)}}}
\newcommand{\IFFTnm}{{\mathrm{IFFT}^{(n,\mp)}}}
\newcommand{\IFFTnmEES}{{\mathrm{IFFT}^{(n,\mp,\rEES)}}}
\newcommand{\IFFTpm}{{\mathrm{IFFT}^{(\Psi,\mp)}}}
\newcommand{\IFFTpmEES}{{\mathrm{IFFT}^{(\Psi,\mp,\rEES)}}}
\newcommand{\FFTni}{{\mathrm{FFT}^{(n,-)}}}
\newcommand{\FFTniEES}{{\mathrm{FFT}^{(n,-,\rEES)}}}
\newcommand{\FFTpi}{{\mathrm{FFT}^{(\Psi,-)}}}
\newcommand{\FFTpiEES}{{\mathrm{FFT}^{(\Psi,-,\rEES)}}}
\newcommand{\IFFTni}{{\mathrm{IFFT}^{(n,-)}}}
\newcommand{\IFFTniEES}{{\mathrm{IFFT}^{(n,-,\rEES)}}}
\newcommand{\IFFTpi}{{\mathrm{IFFT}^{(\Psi,-)}}}
\newcommand{\IFFTpiEES}{{\mathrm{IFFT}^{(\Psi,-,\rEES)}}}
\newcommand{\bgp}{{\bg^{(\Psi)}}}
\newcommand{\bgpEES}{{\bg^{(\Psi,\rEES)}}}
\newcommand{\bgn}{{\bg^{(n)}}}
\newcommand{\bgnEES}{{\bg^{(n,\rEES)}}}
\newcommand{\bsp}{{\hat\bs^{(\Psi)}}}
\newcommand{\bspEES}{{\hat\bs^{(\Psi,\rEES)}}}
\newcommand{\bsn}{{\hat\bs^{(n)}}}
\newcommand{\bsnEES}{{\hat\bs^{(n,\rEES)}}}
\newcommand{\bgpp}{{\bg'^{(\Psi)}}}
\newcommand{\bgppEES}{{\bg'^{(\Psi,\rEES)}}}
\newcommand{\bgpn}{{\bg'^{(n)}}}
\newcommand{\bgpnEES}{{\bg'^{(n,\rEES)}}}
\newcommand{\bspp}{{\hat\bs'^{(\Psi)}}}
\newcommand{\bsppEES}{{\hat\bs'^{(\Psi,\rEES)}}}
\newcommand{\bspn}{{\hat\bs'^{(n)}}}
\newcommand{\bsnpEES}{{\hat\bs'^{(n,\rEES)}}}
\newcommand{\pz}{{(p+\zeta)^3}}
\newcommand{\pzp}{{(p+\zeta_{\Psi})^3}}
\newcommand{\pzn}{{(p+\zeta_n)^3}}
\newcommand{\hsJ}{{<\hs>_{\rNN,J}}}
\newcommand{\hsJp}{{<\hs>_{\rNN,J,\Psi,\zeta_\Psi}}}
\newcommand{\hsJn}{{<\hs>_{\rNN,J,n,\zeta_n}}}
\newcommand{\hsJpzr}{{<\hs>_{\rNN,J,\Psi,0}}}
\newcommand{\hsJnzr}{{<\hs>_{\rNN,J,n,0}}}
\newcommand{\hsinJ}{{\ \ \ \ \ \hs  \in  \hsJ}}
\newcommand{\hsinJp}{{\ \ \ \ \ \hs  \in  \hsJp}}
\newcommand{\hsinJn}{{\ \ \ \ \ \hs  \in  \hsJn}}
\newcommand{\hsinJpzr}{{\ \ \ \ \ \hs  \in  \hsJpzr}}
\newcommand{\hsinJnzr}{{\ \ \ \ \ \hs  \in  \hsJnzr}}
\newcommand{\hspJ}{{<\hs'>_{\rNN,J}}}
\newcommand{\hspJp}{{<\hs'>_{\rNN,J,\Psi,\zeta_\Psi}}}
\newcommand{\hspJn}{{<\hs'>_{\rNN,J,n,\zeta_n}}}
\newcommand{\hspJpzr}{{<\hs'>_{\rNN,J,\Psi,0}}}
\newcommand{\hspJnzr}{{<\hs'>_{\rNN,J,n,0}}}
\newcommand{\hspinJ}{{\ \ \ \ \ \hs'  \in  \hspJ}}
\newcommand{\hspinJp}{{\ \ \ \ \ \hs'  \in  \hspJp}}
\newcommand{\hspinJn}{{\ \ \ \ \ \hs'  \in  \hspJn}}
\newcommand{\hspinJpzr}{{\ \ \ \ \ \hs'  \in  \hspJpzr}}
\newcommand{\hspinJnzr}{{\ \ \ \ \ \hs'  \in  \hspJnzr}}
\newcommand{\hsinpEES}{{\ \ \ \ \hs \in \NFFTpEES}}
\newcommand{\hspinpEES}{{\ \ \ \ \hs' \in \NFFTpEES}}
\newcommand{\hsinnEES}{{\ \ \ \ \hs \in \NFFTnEES}}
\newcommand{\hspinnEES}{{\ \ \ \ \hs' \in \NFFTnEES}}
\newcommand{\hsinn}{{\ \ \ \ \hs \in \NFFTn}}
\newcommand{\hspinn}{{\ \ \ \ \hs' \in \NFFTn}}
\newcommand{\fr}{{\nonumber \ \ \ \ \ \mathrm{From}\ \ \ }}
\newcommand{\vs}{{\vspace{8mm}}}
\newcommand{\ReE}{{{\bf ReE:\ }}}
\newcommand{\ReW}{{{\bf ReW:\ }}}
\newcommand{\ReP}{{{\bf ReP:\ }}}
\newcommand{\ReWG}{{{\bf ReWG:\ }}}
\newcommand{\ReR}{{{\bf ReR:\ }}}
\newcommand{\ReRG}{{{\bf ReRG:\ }}}
\newcommand{\bal}{{\bar{\alpha}}}
\newcommand{\hal}{{\hat{\alpha}}}
\newcommand{\PWerf}{\mathrm{W}^{(\mathrm{erf})}}
\newcommand{\sgn}{\mathrm{sgn}}
\newcommand{\sinhc}{\mathrm{sinhc}}
\newcommand{\rgt}{r_{>}}
\newcommand{\rlt}{r_{<}}
\newcommand{\Derfp}{\mathrm{D}^{(\mathrm{erf})}_p}
\newcommand{\Derfm}{\mathrm{D}^{(\mathrm{erf})}_m}
%----------------------------------------------------------------------------------------
%	TITLE SECTION
%----------------------------------------------------------------------------------------

\newcommand{\horrule}[1]{\rule{\linewidth}{#1}} % Create horizontal rule command with 1 argument of height

\begin{document}
\section{Partial wave expansion of the regularized Coulomb interaction}

Using the more friendly modified Spherical Bessel functions,
\begin{equation}
\begin{split}
\frac{\rerf(\beta|\br-\br'|)}{|\br-\br'|}
&= \frac{2}{\sqrt{\pi}}\int_0^\beta \rd u \re^{-u^2(r^2 + r'^2)}
    \left [\sum_{l=0}^\infty (2l+1) i_{l}(2u^2 r r')P_l\left(\cos(\theta)\right)\right ] \\
&= \frac{2}{\sqrt{\pi}}\sum_{l=0}^\infty (2l+1)P_l\left(\cos(\theta)\right)
      \int_0^\beta \rd u \re^{-u^2(r^2 + r'^2)}i_{l}(2u^2 r r') \\
&= 8\sqrt{\pi}\
      \sum_{l=0}^\infty \sum_{m=-l}^{l} Y_{lm}\left(\cos(\theta),\phi\right)Y_{lm}^*\left(\cos(\theta'),\phi'\right)
      \int_0^\beta \rd u\ \re^{-u^2(r^2 + r'^2)}i_{l}(2u^2 r r') \\
&= \sum_{l=0}^\infty \sum_{m=-l}^{l} Y_{lm}\left(\cos(\theta),\phi\right)Y_{lm}^*\left(\cos(\theta'),\phi'\right)
  \left [ 8 \sqrt{\frac{\pi}{2rr'}}
          \int_0^{\beta\sqrt{2rr'}}\rd x\ \re^{-x^2\left(\frac{r^2 + r'^2}{2rr'}\right)}i_{l}(x^2)
  \right]  \\
&=  \sum_{l=0}^\infty \sum_{m=-l}^{l} Y_{lm}\left(\cos(\theta),\phi\right)Y_{lm}^*\left(\cos(\theta'),\phi'\right)
    \left(\frac{4\pi}{2l+1}\right)\PWerf_l(r,r';\beta)
\end{split}
\end{equation}
where
\begin{equation}
\PWerf_l(r,r';\beta) = \left(\frac{2(2l+1)}{\sqrt{\pi}}\right)\left(\frac{1}{\sqrt{2rr'}}\right)
      \int_0^{\beta\sqrt{2rr'}}\rd x\ \re^{-x^2\left(\frac{r^2 + r'^2}{2rr'}\right)}i_{l}(x^2)
\end{equation}
and
\begin{equation}
\begin{split}
i_0(x) &= \frac{\sinh(x)}{x} = \sinhc(x)\\
i_1(x) &= \frac{x\cosh(x)-\sinh(x)}{x^2} = \frac{\cosh(x)-\sinhc(x)}{x} = \\\\
i_2(x) &= \frac{(x^2+3)\sinh(x)-3x\cosh(x)}{x^3} = \frac{(x^2+3)\sinhc(x)-3\cosh(x)}{x^2} = \\\\
i_3(x) &= \frac{(x^3+15x)\cosh(x)-(6x^2+15)\sinh(x)}{x^4} = \frac{(x^2+15)\cosh(x)-(6x^2+15)\sinhc(x)}{x^3} 
\end{split}
\end{equation}
The symbol $W$ is employed for the interaction as opposed to $\phi$ due neglect of the
prefactor $e^2/4\pi \epsilon_0$ (or just $e^2$ in a.u.) required for the unit of the interaction to be energy.
The factor $4\pi/(2l+1)$ is introduced such that 
$\lim_{\beta\rightarrow\infty}\PWerf_l(r,r';\beta)=\rlt^l/\rgt^{l+1}$, the familar partial wave expansion of
the $1/|\br-\br'|$, where $\rgt = \mathrm{max}(r,r')$ and $\rlt = \mathrm{min}(r,r')$.

The key limits for $W$ are 
\begin{equation}
\begin{split}
\lim_{\beta \rightarrow \infty}\PWerf_l(r,r')
&= \frac{\rlt^l}{\rgt^{l+1}} \\
\lim_{\rlt \rightarrow 0}\PWerf_l(r,r')
&= \frac{\rerf(\beta \rgt)}{\rgt}\delta_{l,0} 
\end{split}
\end{equation}
where $\rgt = \mathrm{max}(r,r')$, $\rlt = \mathrm{min}(r,r')$. The first limit is the partial wave expansion of $1/|\br-\br'|$, the second limit is derived as follows:
\begin{equation}
\begin{split}
\PWerf_l(r,r') &= \left(\frac{2(2l+1)}{\sqrt{\pi}}\right)\left(\frac{1}{\sqrt{2rr'}}\right) \int_0^{\beta\sqrt{2rr'}}\rd x\ \re^{-x^2\left(\frac{r^2 + r'^2}{2rr'}\right)}i_{l}(x^2)\\
&= \left(\frac{2(2l+1)}{\sqrt{\pi}}\right) \int_0^{\beta}\rd x\ \re^{-x^2\left(r^2 + r'^2\right)}i_{l}(2 rr' x^2)\\
\PWerf_l(r,0) 
&= \left(\frac{2(2l+1)}{\sqrt{\pi}}\right) \int_0^{\beta}\rd x\ \re^{-x^2 r^2 }i_{l}(0)\\
&= \left(\frac{2(2l+1)}{\sqrt{\pi}}\right) \int_0^{\beta}\rd x\ \re^{-x^2 r^2 }\delta_{l,0}\\
&= (2l+1) \frac{\rerf(\beta \rgt)}{\rgt}\delta_{l,0} \\
&= \frac{\rerf(\beta \rgt)}{\rgt}\delta_{l,0}
\end{split}
\end{equation}
where the $\lim_{x \rightarrow 0} i_l(x) = \delta_{l,0}$ which can be proved from Eq.1.3, or more generally, from the small argument expansion given by Abramowitz and Stegun's book with leading order term ($i_l(x) \sim x^l$) as x goes to 0. 

The required indefinite integrals are
\begin{eqnarray}
%-----------------------------------------------
\int \rd x\ \re^{-a^2x^2}i_{0}(x^2)&=& 
\frac{\sqrt{\pi}\sgn(x)}{2}\left [a_m \rerfc(a_m |x|) - a_p \rerfc(a_p |x|) \right ] 
- xe^{-a^2x^2} \left [\sinhc(x^2)\right]
\end{eqnarray}

\begin{eqnarray}
%-----------------------------------------------
\int \rd x\ \re^{-a^2x^2}i_{1}(x^2)&=& \frac{1}{6x^3} 
\left [
    \sqrt{\pi} (2 a^2 + 1)a_m x^3 \rerfc(a_m |x|)  - \sqrt{\pi} (2 a^2 - 1)a_p x^3 \rerfc(a_p|x|)
\right. \nonumber \\
& &  \left.  - e^{-(a^2+1) x^2} (2 a^2 (e^{2 x^2} - 1) x^2 + (e^{2 x^2} + 1) x^2 - e^{2 x^2} + 1) 
    \right ] \nonumber \\
%-----------------------------------------------
&=& \frac{1}{6x^3} 
\left [
    \sqrt{\pi} (2 a^2 + 1)a_m x^3 \rerfc(a_m |x|)  - \sqrt{\pi} (2 a^2 - 1)a_p x^3 \rerfc(a_p|x|)
\right. \nonumber \\
& &  \left.  -2e^{-a^2x^2} \left [ (2a^2x^2-1)\sinh(x^2) + x^2\cosh(x^2)\right ]
    \right ] \nonumber \\
%-----------------------------------------------
&=& \frac{\sqrt{\pi}}{6}\left [ (2 a^2 + 1)a_m \rerfc(a_m |x|)  - (2 a^2 - 1)a_p \rerfc(a_p|x|) \right ]
\nonumber \\
& &  -\frac{e^{-a^2x^2}}{3x^3}\left [ (2a^2x^2-1)\sinh(x^2) + x^2\cosh(x^2)\right ] \nonumber \\
%-----------------------------------------------
&=& \frac{\sqrt{\pi}}{6}\left [ (2 a^2 + 1)a_m \rerfc(a_m |x|)  - (2 a^2 - 1)a_p \rerfc(a_p|x|) 
   \right ] \nonumber \\
& &  -\frac{e^{-a^2x^2}}{3x}\left [ (2a^2x^2-1)\sinhc(x^2) + \cosh(x^2)\right ]
%-----------------------------------------------
\end{eqnarray}

\begin{eqnarray}
%-----------------------------------------------
  \int \rd x\ \re^{-a^2x^2}i_{2}(x^2) &=&
  \frac{1}{10 x^5}\left [
 -\sqrt{\pi} (4 a^4 + 2 a^2 - 1) x^4 a_m|x| \rerf(a_m|x|) 
 +\sqrt{\pi} (4 a^4 - 2 a^2 - 1) x^4 a_p|x| \rerf(a_p|x|) \right .\nonumber \\
& & \left .  \ \ \ \ \ \
 +\sqrt{\pi} x^4 (2 a^2 (a_m|x| + a_p|x|) - a_m|x| + a_p|x| + 4 a^4 (a_m|x| - a_p|x|) )
 \right . \nonumber \\
 & & \left . \ \ \ \ \ \
 + e^{-(a^2 - 1) x^2} ((2 a^2 + 3) x^2 + (-4 a^4 - 2 a^2 + 1) x^4 - 3)  \right. \nonumber \\
 & & \left .  \ \ \ \ \ \
 + e^{-(a^2 + 1) x^2} ((3 - 2 a^2) x^2 + (4 a^4 - 2 a^2 - 1) x^4 + 3) \right ]
  \nonumber   \\
%-----------------------------------------------
&=&
\frac{\mathrm{sgn}(x)}{10}\left [
- \sqrt{\pi} (4 a^4 + 2 a^2 - 1) a_m \rerf(a_m|x|) 
 +\sqrt{\pi} (4 a^4 - 2 a^2 - 1) a_p \rerf(a_p|x|) \right . \nonumber \\
& &\left . \ \ \ \ \ \
 +\sqrt{\pi} (2 a^2 (a_m + a_p) - a_m + a_p + 4 a^4 (a_m - a_p) )
 \right ] \nonumber \\
 &+& \frac{1}{5x^5}\left [
 + e^{-a^2 x^2} (2 a^2  x^2 \sinh(x^2) + (-4 a^4 + 1)x^4\sinh(x^2) + (3x^2-2 a^2  x^4)\cosh(x^2) 
  - 3\sinh(x^2))  \right ] \nonumber  \\
%-----------------------------------------------
&=&
\frac{\sqrt{\pi} \mathrm{sgn}(x)}{10}\left [
 (4 a^4 + 2 a^2 - 1) a_m \rerfc(a_m|x|) 
 -(4 a^4 - 2 a^2 - 1) a_p \rerfc(a_p|x|) \right ] \nonumber \\
 &+& \frac{e^{-a^2 x^2} }{5x^3}\left [
 ( (1-4 a^4)x^4 + 2 a^2 x^2 -3)\sinhc(x^2) + (3 - 2 a^2 x^2)\cosh(x^2) +4 a^4x^4
  \right ] \nonumber \\
 &-& \frac{4xa^4e^{-a^2 x^2} }{5}
\end{eqnarray}
Here, the integrals are presented in a form in which is easier to check by differentiating wrt to x and
ensuring that the limit $x\rightarrow 0$ is well defined.

\begin{eqnarray}
\int \rd x\ \re^{-a^2x^2}i_{3}(x^2) &=& \nonumber \\
\end{eqnarray}

where
\begin{equation}
\begin{split}
a_m &= \sqrt{a^2-1} \\
a_p &= \sqrt{a^2+1} 
\end{split}
\end{equation}

In order to evalute the partial waves, we insert the definition of the a's
\begin{equation}
\begin{split}
a   &=  \frac{\sqrt{\rgt^2+\rlt^2}}{\sqrt{2\rgt\rlt}} \\
a_m &=  \frac{\rgt - \rlt}{\sqrt{2\rgt\rlt}} \\
a_p &=  \frac{\rgt + \rlt}{\sqrt{2\rgt\rlt}} 
\end{split}
\end{equation}
where as aboave $\rgt = \mathrm{max}(r,r')$ and $\rlt = \mathrm{min}(r,r')$. 
Furtheremore, since the upper and lower limits of the integral are positive semi-definite ($x \geq 0$),
we only need to be careful
about the absolute value in the evalution of $a_p$ and $a_m$ and not in $x$ which is evaluted at the limits.
Hence $\sgn(x)\equiv 1$ and $|x|\equiv x$.  It is nontheless
convenient to introduce $\rgt$ and $\rlt$ everywhere rather have mixed notation of $r,r'\rgt,\rlt$

Using the simplified form of the integrals above and the definition of the a's in terms of $\rgt$ and $\rlt$,
the coefficients of the partial wave expansion of the regularized / cutoff Coulomb interaction can be written in
a form in which the limits $\rlt\rightarrow 0$, $\beta\rightarrow \infty$ and $\rgt\rightarrow\rlt$ 
can be evaluated by easily.  For convenience functions,
\begin{equation}
\begin{split}
 \Derfp(\rlt,\rgt;\beta) &=\frac{1}{2}\left[\rerf(\beta (\rgt+\rlt))+\rerf(\beta (\rgt-\rlt))\right]\\
 \Derfm(\rlt,\rgt;\beta) &=\frac{1}{2}\left[\rerf(\beta (\rgt+\rlt))-\rerf(\beta (\rgt-\rlt))\right] \\
 G_H(\rlt,\rgt,\beta) &= \left(\frac{2\beta}{\sqrt{\pi}}\right) e^{-\beta^2(\rgt^2+\rlt^2)}
\end{split}
\end{equation}
are defined. The partial wave are
\begin{eqnarray}
\PWerf_0(r,r';\beta) &=& 
 \left[\Derfp(\rlt,\rgt;\beta)\right]\left[\frac{1}{\rgt}\right] 
+\left[\Derfm(\rlt,\rgt;\beta)-\rlt G_H(\rlt,\rgt,\beta)\right]\left[\frac{1}{\rlt}\right] \\
&-& G_H(\rlt,\rgt,\beta)
   \left [\sinhc(2\beta^2 \rgt\rlt)-1\right ]
\nonumber
\end{eqnarray}
%
\begin{eqnarray}
  \PWerf_1(r,r';\beta) &=&
   \left[\Derfp(\rlt,\rgt;\beta) -\rgt G_H(\rlt,\rgt,\beta)\right]
   \left [\frac{\rlt}{\rgt^2} \right] \nonumber\\
&+&  \left [\Derfm(\rlt,\rgt;\beta)-\rlt G_H(\rlt,\rgt,\beta)\right]
     \left [\frac{\rgt}{\rlt^2} \right] \\
&-&  G_H(\rlt,\rgt,\beta)
    \left [\frac{ (2\beta^2(\rgt^2+\rlt^2)-1)\sinhc(2\beta^2\rgt\rlt) 
          + \cosh(2\beta^2\rgt\rlt)- 2\beta^2(\rgt^2+\rlt^2)}{2\beta^2\rgt\rlt}\right ]
\nonumber 
\end{eqnarray}
%
\begin{eqnarray}
\PWerf_2(r,r';\beta) &=&  \nonumber \\
\end{eqnarray}
%
\begin{eqnarray}
\PWerf_3(r,r';\beta) &=&  \nonumber \\
\end{eqnarray}

\clearpage
The alternative form of the integral is 
\begin{eqnarray}
\int \rd x\ \re^{-a^2x^2}i_{1}(b^2x^2)&=& \frac{1}{6b^4} 
\left [
  \sqrt{\pi}\left (-(a^2 + a_m^2) a_p \rerfc(x a_p)+(a^2 + a_p^2) a_m \rerfc(x a_m) \right) \right . \nonumber \\
& & \left .
- \frac{2b^2e^{-x^2 a^2}}{x} \left ((2 a^2 x^2 - 1)\frac{\sinh(b^2x^2)}{b^2x^2} + \cosh(b^2x^2) \right)
\right ]
\end{eqnarray}

where
\begin{equation}
\begin{split}
a   &=  \sqrt{\rgt^2+\rlt^2}\\
b   &=  \sqrt{2\rgt\rlt} \\
a_m &=  \rgt-\rlt \\
a_p &=  \rgt+\rlt \\
\end{split}
\end{equation}
To evaluate the partial wave, the limits are simple, $0$ and $\beta$.

\clearpage

\begin{eqnarray}
I_2(x) &=& 
\frac{\sqrt{\pi}}{10}\left [
 (4 a^4 + 2 a^2 - 1) a_m \rerfc(a_m x) 
 -(4 a^4 - 2 a^2 - 1) a_p \rerfc(a_p x) \right ] \nonumber \\
 &+& \frac{e^{-a^2 x^2} }{5x^3}\left [
 ( (1-4 a^4)x^4 + 2 a^2 x^2 -3)\sinhc(x^2) + (3 - 2 a^2 x^2)\cosh(x^2) +4 a^4x^4
  \right ] \nonumber \\
 &-& \frac{4xa^4e^{-a^2 x^2} }{5} \nonumber
\end{eqnarray}

\begin{eqnarray}
 \frac{10}{\sqrt{\pi}}\frac{I_2(\beta \sqrt{2\rgt\rlt})}{\sqrt{2\rgt\rlt}}
 &=& \left\{\frac{\sqrt{\pi}}{10}\left [
 (4 a^4 + 2 a^2 - 1) a_m \rerfc(a_m x) 
 -(4 a^4 - 2 a^2 - 1) a_p \rerfc(a_p x) \right ]  \right. \nonumber \\
 %\left. &+& \frac{e^{-a^2 x^2} }{5x^3}\left [
 %( (1-4 a^4)x^4 + 2 a^2 x^2 -3)\sinhc(x^2) + (3 - 2 a^2 x^2)\cosh(x^2) +4 %a^4x^4
%  \right ] \nonumber \right. \\
%\left.  &-& \frac{4xa^4e^{-a^2 x^2} }{5} \right\} \frac{1}{\sqrt{2\rgt\rlt}}\nonumber \\
\left. &\ &\right\} \frac{1}{\sqrt{2\rgt\rlt}}
\end{eqnarray}


\end{document}
