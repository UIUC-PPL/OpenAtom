%==================================================================
%cccccccccccccccccccccccccccccccccccccccccccccccccccccccccccccccccc
%==================================================================

\documentclass[12pt,titlepage]{article}
\usepackage[text={6.5in,8.9in},centering]{geometry}

\newcommand{\openatom}{\textsc{OpenAtom}}


% title and date
\title{\openatom Manual: Modern Simulation Methods Applied to Chemistry, Physics and Biology}
\date{\today}

\begin{document}

% make title and table of contents band break for page
\maketitle
\tableofcontents
\newpage

%==================================================================
% Old title page
%==================================================================
%\leavevmode\vfill
%\begin{center}
%\huge
%\textsc{OpenAtom} Manual: \\
%$\ $ \\
%\Large
%Modern Simulation Methods Applied to Chemistry, Physics and Biology
%\end{center}
%\vfill
%\newpage

%==================================================================
%==================================================================
\hrulefill
\section{\bf Introduction}

There are five types of files from which the \textsc{OpenAtom}
script-language interface reads commands, the
simulation setup file, the system setup file, molecular parameter/topology
files, the potential energy/pseudopotential parameter files and the
parallel decomposition parameter file. Coordinate and electronic state input files do 
not contain commands but simply free format data. 

In the simulation setup files, commands that drive a given simulation
are stated. For example, run 300 time steps of Car-Parrinello {\em ab initio}
molecular dynamics. In the system setup file, the 
system is described. For example, 300 water molecules, two peptides,
three counter ions, thirty Kohn-Sham states etc.  Topology files
contain information about the molecular connectivity and potential
energy/pseudopotential files contain information about the
interactions. The parallel decomposition
input file contains all the information about the parallel decomposition.

The idea behind the file division is that one often modifies the
simulation commands, occasionally the system setup commands but seldom the 
topology, potential energy,pseudopotential or parallel decomposition
information. 
The five file types help organization and transferability. For
example, many simulation command files can drive the same system setup file, 
molecular parameter/topology files, potential energy/pseudopotential
parameter and parallel decomposition files.

There are three levels of grouping for the \textsc{OpenAtom} commands, 
meta-keywords, keywords and key-arguments.
The first level, the meta-keywords, sort keywords into naturally
connected groups. The second level, the keywords, are specific
commands to the computer program. The third level, key-arguments,
are the arguments to the keywords.  The syntax looks like: \\ 
\hspace*{0.5in} \~{}meta\_keyword1[ \textbackslash{}keyword1\{keyarg1\}
\textbackslash{}keyword2\{keyarg2\} ]. \\
\hspace*{0.5in} \~{}meta\_keyword2[ \textbackslash{}keyword1\{keyarg1\}
\textbackslash{}keyword2\{keyarg2\} ]. \\
The commands are case insensitive and can be specified in any order.

The present release of the code concentrates on the fine grained
parallel Car-Parrinello molecular dynamics (CPAIMD) computations 
and scaling results are given in IBM J. Res. Dev. {\bf 52} (2007). 
However, all the 
bio-widget input parameters are supported and biomolecules can be
built. Some of the biomolecular functionality is not yet implemented
in the source. The QM/MM extension is a work in progress.

In order to run the code type: \\
\hspace*{1.5in} {\bf \textsc{OpenAtom}Machine.x paraInfo.in simInput.in} \\
where machine is the machine-type, simInput.in is the simulation input
file and paraInfo.in is the parallel decomposition input file. 
All file names are arbitrary.

{\bf Warning:} The code will warn the user and stop the run if it isn't happy
with the input. The designers felt that {\bf not} running a
simulation was better than taking potentially inappropriate input 
and forging ahead. 
Also, the code doesn't like to overwrite files because the designers
have overwritten one too many useful files themselves and thought,  \\
\hspace*{1.25in}  A file that big?! \\
\hspace*{1.25in}  It might be very useful,\\
\hspace*{1.25in}  But now it is gone.\\
Finally, many of the error messages are sometimes written in colloquial American English, 
a product of too many late nights writing code. A language option has
not yet been implemented. The advisor may have to explain what
``Dude'' means.  However, the code does have a lot of error
checking because there is nothing worse than \\
\hspace*{1.25in}    Wind catches lily, \\
\hspace*{1.25in}    Scatt'ring petals to the wind:\\
\hspace*{1.25in}    Segmentation fault.\\
which leads one to believe \\
\hspace*{1.25in}  Errors have occurred. \\
\hspace*{1.25in}  I can't tell where or why.\\
\hspace*{1.25in}  Lazy programmers.\\
This manual was, of course, also a late night accomplishment.

\newpage
%==================================================================
%==================================================================
\hrulefill
%% instruction for Martins-Troullier type pseudopotential
%
%\documentclass[letterpaper,12pt]{article}
%
%\usepackage{helvet}
%\usepackage{courier}
%
%
%
%\textwidth = 6.5 in
%\textheight = 9 in
%\oddsidemargin = 0.0 in
%\evensidemargin = 0.0 in
%\topmargin = 0.0 in
%\headheight = 0.0 in
%\headsep = 0.0 in
%\parskip = 0.2in
%\parindent = 0.0in
%
%
%\title{\fontfamily{phv}\selectfont{Converting pseudopotentials for OpenAtom}}
%%\author{\fontfamily{phv}\selectfont{Minjung Kim}}
%\date{}
%
%\begin{document}
%\maketitle

\section{Converting pseudopotentials for OpenAtom}


At this moment (May 2015),
\begin{itemize}
\item OpenAtom only supports norm-conserving pseudopotentials.
\item The converter works for CPMD pseudopotentials. You may want to download CPMD pseudopotential library (visit \verb+http://cpmd.org/download+.)
\end{itemize}
--- Minjung Kim

%%%%%%%%%%%%%%%%% MARTINS-TROULLIER 
\subsection{Martins-Troullier type pseudopotentials}


\noindent
\underline{ {\bf Source codes }(provided): }
	\begin{itemize}
	\item make{\_}mt{\_}pot.f
	\item piny{\_}make{\_}grid{\_}mt.f
	\end{itemize}
	
\vskip 12pt
\noindent
\underline{ {\bf Input files:} }
	\begin{itemize}
	\item potential file (e.g., Zn{\_}mt{\_}pot)
	\item wavefunction file (e.g., Zn{\_}mt{\_}wfn)
	\item PI{\_}MD.MAKE : This file contains input information needed to run piny{\_}make{\_}grid{\_}mt.f
	\end{itemize}

\vskip 12pt
\noindent
\underline{ {\bf Instruction:} }

\begin{enumerate}
\item Open CPMD pseudopotential file you wish to convert (e.g., Zn\_MT\_PBE\_SEMI.psp)
\item  Copy potential and wavefunction data from CPMD pseudopotential file (right after \verb+&POTENTIAL+ and \verb+&WAVEFUNCTION+) to potential (Zn{\_}mt{\_}pot) and wavefunction file (Zn\_mt\_wfn)
\item Edit make{\_}mt{\_}pot.f

        \begin{itemize}
	\item Enter the names of your potential and wavefunction files into the appropriate ``open" statement (line 3 and 4)
	\item Enter the names of your angular momentum channel files  you wish to create into the ``open" statement. 
	For example, if you have 3 angular momentum channels ($s, p, d$), then Zn{\_}l0, Zn{\_}l1, Zn{\_}l2
         \end{itemize}

\item Compile make{\_}mt{\_}pot.f and run. It will create Zn{\_}l0, Zn{\_}l1, Zn{\_}l2 files.

\item Edit PI{\_}MD.MAKE\\ The following informations are required:

	\begin{itemize}
	\item npts rmax lmax
	\item ZV
	\item alpha1 C1
	\item alpha2 C2
	\item l=0 file name 
	\item l=1 file name 
	\item l=2 file name 
	\item...
	\item l=lmax file name
	\end{itemize}
	
	npts: number of points on the ouput radial grid\\
	rmax: the largest r value you on the radial grid\\
	lmax: the highest angular momentum channel\\
	ZV: the valence charge\\
	alpha1, alpha2, C1, C2: parameters that determine the long-range part of the pseudopotential, assumed to be:\\
	
	-C1*erf(alpha1*r)/r - C2*erf(alpha2*r)/r\\
	
	We've been using alpha1=1.0 C1=1.0 and C2=0.0 so that the long-range part is -erf(r)/r

\item Compile piny{\_}make{\_}grid{\_}mt.f and run. It will create OpenAtom pseudopotential file.
\end{enumerate}



%%%%%%%%%%%%%%%%% Goedecker-Hutter type 
\subsection{Goedecker-Hutter type pseudopotential}

\noindent
\underline{ {\bf Source code }(provided): }
	\begin{itemize}
	\item make\_goedecker.c
	\end{itemize}

\vskip 12pt
\noindent
\underline{ {\bf Input file:} }
	\begin{itemize}
	\item input file (e.g., Cl\_SG\_BLYP\_INPUT -- see the example in the directory)
	\end{itemize}
	
	
\vskip 12pt
\noindent
\underline{ {\bf Instruction:} }

\begin{enumerate}
\item Create an input file which includes the following information:

First line -comments\\
Zion\\
ZV (valence electrons)\\
LMAX\\
rloc\\
number of C, C1, C2, C3, C4\\
Number of s channels\\
rs\\
h0(1,1)\\
h0(2,2)\\
Number of p channels\\
rp\\
h1(1,1)\\
Number of d channels\\
Number of f channels\\

You can get these numbers from CPMD PP file. Or look up PRB \textbf{58}, 3641 (1998) paper. All coefficients are tabulated.

\item Run \verb+make_goedecker.c+. You are prompted for the name of the input file, and it will create \verb+PSEUDO.OUT+.


\end{enumerate}




%\end{document}


\newpage
%==================================================================
%==================================================================
\hrulefill
%% instruction for Martins-Troullier type pseudopotential
%
%\documentclass[letterpaper,12pt]{article}
%
%\usepackage{helvet}
%\usepackage{courier}
%
%\textwidth = 6.5 in
%\textheight = 9 in
%\oddsidemargin = 0.0 in
%\evensidemargin = 0.0 in
%\topmargin = 0.0 in
%\headheight = 0.0 in
%\headsep = 0.0 in
%\parskip = 0.2in
%\parindent = 0.0in
%
%\newcommand{\openatom}{\textsc{OpenAtom}}
%\renewcommand\labelitemi{\textendash}
%
%
%\title{\fontfamily{phv}\selectfont{\openatom: Get started}}
%%\author{\fontfamily{phv}\selectfont{Minjung Kim}}
%\date{}
%
%\begin{document}
%\maketitle

\section{\textsc{OpenAtom}: Getting started}

Before running any type of ``CP'' (which means quantum in \openatom \ land) simulations supported by \openatom \ package, you first need to obtain minimized wavefunction files by performing \verb+cp_wave_min+.

This document explains how to make input files for \verb+cp_wave_min+ with a system of one water (H$_2$O) molecule.


--Minjung Kim

%%%%%%%%%%%%%%%%% Compilation
%\subsection{Compilation}

%%%%%%%%%%%%%%%%% Input files

\subsection{Input files}

There are 7 different types of input files to run \openatom. 

\subsubsection{Simulation keyword file}

Let's create a file named \verb+water.input+. The name of the input file is your choice. This file includes general information of the simulation such as, what  type of simulation to perform, where to write the output files, where to find other input files, etc. 

The simulation keyword file contains 9 subsections (i.e., meta-keywords):
\begin{verbatim}
~sim_list_def[ ]
~sim_cp_def[ ]
~sim_gen_def[ ]
~sim_class_PE_def[ ]
~sim_run_def[ ]
~sim_nhc_def[ ]
~sim_vol_def[ ]
~sim_write_def[ ]
~sim_pimd_def[ ]
\end{verbatim}

It is not required to use all meta-keywords in the input file. For example, if you are not running your calculation with multiple beads, \verb+sim_pimd_def[]+ meta-keyword is unnecessary. 

Each meta-keyword requires keywords and key-arguments. Complete information of keywords and key-arguments is found in \openatom \ website. This document explains a few keywords and its key-arguments that are necessary to run \verb+cp_wave_min+.
Here is the example of \verb+water.input+:

\begin{verbatim}
~sim_gen_def[
\simulation_typ{cp_wave_min}
\num_time_step{100000}
\restart_type{initial}
]
 
~sim_cp_def[
\cp_minimize_typ{min_cg}
\cp_restart_type{gen_wave}
]

~sim_run_def[
\cp_min_tol{0.001}
]

~sim_vol_def[
\periodicity{3}
]

~sim_write_def[
\write_screen_freq{3}
\write_dump_freq{100}
\in_restart_file{water.coords_initial}
\mol_set_file{water.set}
\sim_name{water}
\write_binary_cp_coef{off_gzip}
]
\end{verbatim}

Short explanation for each keyword \& argument:
\begin{itemize}
\item Simulation type is \verb+cp_wave_min+, which means minimizing wavefunction coefficients.
\item It performs 100000 iterations. After 100000 iteration, it stops even if it does not converge.
\item Restart type is \verb+initial+, which means it starts from scratch.
\item \verb+min_cg+ uses the conjugate gradient method for minimization.
\item \verb+gen_wave+ stands for generating wavefunction coefficients.
\item Once the force of wavefunction reaches to the \verb+cp_min_tol+ value, it stops the wavefunction minimization.
\item \verb+periodicity+ indicates the boundary condition. 3 means it is fully periodic.
\item \verb+write_screen_freq+ defines the frequency of writing information to an output file. 
\item \verb+write_dump_freq+ determines how frequently write the wavefunction coefficients to \verb+STATES_OUT+ directory. 
\item \verb+in_restart_file+ reads the name of coordinate input file. \verb+water.coords_initial+ will be read from \verb+ATOM_COORDS_IN+ directory.
\item \verb+mol_set_file+ reads the setup file named \verb+water.set+. 
\item \verb+sim_name+ sets the simulation name. In this example, \verb+water+ file will be generated at the end of the simulation.
\item \verb+off_gzip+ means the wavefunction coefficients are not binary and gzipped.
\end{itemize}


%%%%%%%%%%%%%%%%%%%%%%%%%%%%%%%%%%%%%%SETUP file
\subsubsection{Setup file}
In the input file, keyword \verb+\mol_set_file+ has a key-argument \verb+water.set+. This is a setup file. It gives the information of total number of electrons, types of of molecules (or atoms), pseudopotential database, etc.

The setup file contains 4 subsections (=meta-keywords):
\begin{verbatim}
~wavefunc_def[ ]
~molecule_def[ ]
~data_base_def[ ]
~bond_free_def[ ]
\end{verbatim}

The first three meta-keywords are necessary. For ``CP'' simulations, we do not need the last meta-keyword. Below is what \verb+water.set+ may look like:

\begin{verbatim}
~wavefunc_def[
\nstate_up{4}\nstate_dn{4}\cp_nhc_opt{none}
 ]

~molecule_def[
\mol_name{Hydrogen}\mol_parm_file{./DATABASE/H.parm}
\num_mol{2}\mol_index{1}\mol_opt_nhc{mass_mol}
]

~molecule_def[
\mol_name{Oxygen}\mol_parm_file{./DATABASE/O.parm}
\num_mol{1}\mol_index{2}\mol_opt_nhc{mass_mol}
]

~data_base_def[
\inter_file{./water.inter}
\vps_file{./water.vps}
]
\end{verbatim}

\begin{itemize}
\item \verb+wavefunc_def+ defines the number of electrons. The number of states for spin-up and spin-down has to be the same. 

\item \verb+molecule_def+ defines the name of the molecule and specify the parameter file (\verb+.parm+ files). In quantum simulations, molecule should be the individual atom.

\item \verb+data_base_def+ defines the name of interaction file (\verb+water.inter+) and pseudopotential file (\verb+water.vps+).  

\end{itemize}

% Coordinates
\subsubsection{Coordinate input file}
This file includes the information of atom coordinates and the size of the simulation cell. The name of the atom coordinate file (e.g., \verb+water.coords_initial+) is defined in input file (see subsection above).

The coordinate file includes three parts:

[number of atoms] [number of beads] [number of path]\\
coordinates (in \AA)\\
simulation cell information (in \AA)\\

In this example, \verb+water.coords_initial+ may look like below:

\begin{verbatim}
3 1 1
0.757 0.586 0.0
-0.757 0.586 0.0
0.000 0.000 0.0
10 0 0
0 10 0
0 0 10
\end{verbatim}

Make sure that the coordinates must follow the order defined in \verb+water.set+ file with \verb+\mol_index+ keyword, i.e., the first two coordinates for H, and the third coordinate for O. 

% PARALLEL DECOMPOSITION PARAMETERS (CHARM PARAMETERS)
\subsubsection{Charm parameter file}

The charm parameter file includes all options related to charm++. It can be an empty file. Usually, that is the best way to start. If  \openatom \ complains about some key-arguments, change it accordingly. In our example, we will name this file as \verb+cpaimd_config+.


% Potential parameter keywords file
\subsubsection{Potential parameter files}

In the setup file (\verb+water.set+), we have specified two files: \verb+water.inter+ and \verb+water.vps+. For ``CP'' calculations (remember in \openatom \ world, CP means quantum), \verb+water.inter+ is irrelevant. However, if this file does not exist, the code will complain, so it is necessary to create this file as well. \verb+water.vps+ indicates types of the pseudopotential, names of the pseudopotential file, the number of angular momentum, and local component of the pseudopotential. Here are examples of \verb+water.inter+ and \verb+water.vps+.

\verb+water.inter+ file:

\begin{verbatim}
~inter_parm[\atom1{H}\atom2{H}\pot_type{null}\min_dist{0.1}\max_dist{12.9}]
~inter_parm[\atom1{O}\atom2{H}\pot_type{null}\min_dist{0.1}\max_dist{12.9}]
~inter_parm[\atom1{O}\atom2{O}\pot_type{null}\min_dist{0.1}\max_dist{12.9}]
\end{verbatim}

\verb+water.vps+ file:

\begin{verbatim}
~PSEUDO_PARM[\ATOM1{H}\VPS_TYP{LOC}\N_ANG{0}\LOC_OPT{0}
\VPS_FILE{./DATABASE/H_BLYP_PP30.pseud}]

~PSEUDO_PARM[\ATOM1{O}\VPS_TYP{KB}\N_ANG{1}\LOC_OPT{1}
 \VPS_FILE{./DATABASE/O_BLYP_PP30.pseud}]

\end{verbatim}

% Topology keywords file (.parm file)
\subsubsection{Topology keywords file}

In setup file, the topology keywords file is defined in \verb+\mol_parm_file+ keyword for each molecule. In this file, we specify which atoms consist of the molecule (in our case, atom is the molecule). The example of these files for our water system is below:

\verb+H.parm+ file:

\begin{verbatim}
~MOLECULE_NAME_DEF[\MOLECULE_NAME{Hydrogen}\NATOM{1}]

~ATOM_DEF[\ATOM_TYP{H}\ATOM_IND{1}\MASS{1.0}\CHARGE{1.0}
\cp_valence_up{0}\cp_valence_dn{0}\cp_atom{yes}]
\end{verbatim}

\verb+O.parm+ file:

\begin{verbatim}
~MOLECULE_NAME_DEF[\MOLECULE_NAME{Oxygen}\NATOM{1}]

~ATOM_DEF[\ATOM_TYP{O}\ATOM_IND{1}\MASS{16.0}\CHARGE{6.0}
\cp_valence_up{4}\cp_valence_dn{4}\cp_atom{yes}]
\end{verbatim}

% Pseudopotential files
\subsubsection{Pseudopotential files}
The name of the pseudopotential files are indicated in potential parameter file (\verb+water.vps+). Generating pseudopotential for \openatom \ can be found in the \openatom \ website.



%\end{document}


\newpage
%==================================================================
%==================================================================
\hrulefill
\section{\bf Simulation Keyword Dictionary} 

The following commands maybe specified in the simulation setup file:
\begin{enumerate}
\item {\bf \~{}sim\_gen\_def} [16 keywords]
\item {\bf \~{}sim\_run\_def} [16 keywords]
\item {\bf \~{}sim\_nhc\_def} [28 keywords]
\item {\bf \~{}sim\_write\_def} [27 keywords]
\item {\bf \~{}sim\_list\_def} [14 keywords]
\item {\bf \~{}sim\_class\_PE\_def} [25 keywords]
\item {\bf \~{}sim\_vol\_def} [4 keywords]
\item {\bf \~{}sim\_cp\_def} [29 keywords]
\item {\bf \~{}sim\_pimd\_def} [10 keywords]
\end{enumerate}
A complete list of all commands employed including default
values are placed in the file:
\~{}sim\_write\_def[ \textbackslash{}sim\_name\_def\{sim\_input.out\} ]. 


\newpage
%==================================================================
\subsection*{\bf \~{}sim\_gen\_def [16 keywords]}

%-------------------------------------------------------------------
\begin{enumerate}
 \vspace{0.15in}
 \item \textbackslash{}simulation\_typ\{{\bf cp},cp\_pimd,...\} : \\
     This describes the type of simulation being performed.
     \begin{itemize}
      \item cp\_wave\_min : Minimize/optimize the Kohn-Sham wave functions (fixed nuclei)
              If you don't have initial wave functions, you have to
	      use the gen\_wave option below.\\
      \item cp\_wave\_min\_pimd : Minimize/optimize the Kohn-Sham wave functions
		   over all the beads in preparation for a path
		   integral run (PIMD) with fixed nuclei.  If you need
		   starting wave functions for all beads, you need gen\_wave.\\
       \item cp : Car-Parrinello molecular dynamics (MD)\\
       \item cp\_pimd : Car-Parinello path integral MD over all the beads\\
       \item bomd : Born-Oppenheimer molecular dynamics\\
       \item bomd\_pimd : Born-Oppenheimer molecular dynamics for beads in path integral MD\\
       \item 
{\it
--- planned for the near future---\\
cp\_min : Minimize nuclei on the Born-Oppenheimer surface (provided by DFT)
}
 \end{itemize}

 \vspace{0.15in} 
 \item  \textbackslash{}ensemble\_typ\{{\bf nve},nvt,...\} : \\
Contolls the type of statistical ensemble used in the MD simulation
\begin{itemize}
\item nve : Run a (N,V,E) microcanonical MD simulation\\
\item nvt : Run a (N,V,T) canonical simulation, temperature is controlled by 
      temperature keyword below\\
\item {\it
--- planned for the near future ---\\
npt\_i : (N,P,T) ensemble with volume scaling only (e.g., for liquids)\\
npt\_f : (N,P,T) ensemble with full unit cell scaling  (e.g., solids
        with shear modes)
}
\end{itemize}

 \vspace{0.15in} 
 \item  \textbackslash{}restart\_type\{{\bf initial},
             restart\_pos,restart\_posvel,restart\_all\} : \\
     Option controlling how the coordinate input file is read in:
     \begin{itemize}
       \item  initial -- Start using {\it initial} format 
                                        with atomic positions in Angstroms.
       \item  restart\_pos -- Restart only the atomic positions.
                    Velocities and thermostat velocites (if used) are 
                    sampled from a Maxwell distribution.
       \item  restart\_posvel -- Restart both atmoic position 
                     and atomic velocities.  Thermostat velocities (if used) 
                     are sampled from a Maxwell distribution.
       \item restart\_all -- Restart atomic positions, 
                     velocities, and thermostat velocities (if used).
     \end{itemize}



 \vspace{0.15in} 
 \item  \textbackslash{}num\_time\_step\{0\} : \\
Total number of time steps (or iterations) of the simulation.  Zero is
an error, it must be a positive integer.  

 \vspace{0.15in} 
 \item   \textbackslash{}time\_step\{1\} : \\
Units of femtoseconds.  This is the fundamental or smallest time
step. If no multiple time step integration is used, then this is the
time step.  For example, in a cp simulation, this is the time step for
each CP step.  In a minimization process, this controls the size of
the minimization step.


 \vspace{0.15in} 
 \item   \textbackslash{}temperature\{300\} : \\
Tempeature for nuclei during MD simulations at finite temperature.
This is the temperature used for velocity rescaling, velocity
resampling, thermostatting, etc. The units are Kelvin.


 \vspace{0.15in} 
 \item   \textbackslash{}pressure\{0\} : \\
Pressure of system in bar. Only used for simulations in the NPT
ensembles.



 \vspace{0.15in} 
{\it

 \item  --- planned for the near future --- \\
 \textbackslash{}minimize\_typ\{min\_std,min\_cg,min\_diis\} : \\
     Perform a minimization run using the method specified in curly brackets:
     \begin{itemize}
        \item min\_std -- Steepest descent.
        \item min\_cg -- Conjugate gradient.
        \item min\_diis -- Direct inversion in the 
                                          iterative subspace.
     \end{itemize}
}

\end{enumerate}


\newpage
%-------------------------------------------------------------------
\subsection*{\bf \~{}sim\_run\_def [16 keywords]}

\begin{enumerate}
 \vspace{0.15in}
 \item   \textbackslash{}init\_resmp\_atm\_vel\{on,{\bf off}\} : \\
     If turned on, atomic velocities are initially sampled/resampled from a 
     Maxwell distribution.


 \vspace{0.15in}
 \item   \textbackslash{}resmpl\_frq\_atm\_vel\{0\} : \\
     Atomic velocities are resampled with a frequency equal to the number 
     in the curly brackets.  A zero indicates no resampling is to be done.


 \vspace{0.15in}
 \item   \textbackslash{}init\_rescale\_atm\_vel\{on,{\bf off}\} : \\
    If turned on, atomic velocities are initially rescaled to preset 
    temperature.

 \vspace{0.15in}
 \item   \textbackslash{}rescale\_frq\_atm\_vel\{0\} : \\
    Atomic velocities are rescaled to desired temperature
    with a frequency equal to the number in the
    curly brackets.  A zero indicates no resampling is to be done.

 \vspace{0.15in} 
 \item   \textbackslash{}init\_rescale\_atm\_nhc\{on,{\bf off}\} : \\
    If turned on, atomic thermostat velocities are initially rescaled to 
    preset temperature.

 \vspace{0.15in} 
 \item   \textbackslash{}zero\_com\_vel\{yes,{\bf no}\} : \\
    If set to yes, the center of mass velocity is initially set to 0.

 \vspace{0.15in} 
 \item   \textbackslash{}respa\_steps\_lrf\{0\} : \\
    Number of short time steps to be taken between long range force updates.
    A zero indicates long range forces are updated every step.

 \vspace{0.15in} 
 \item   \textbackslash{}respa\_rheal\{1\} : \\
    If \textbackslash{}respa\_steps\_lrf is not zero, then the long range 
    and short range forces are both switched off in space with a switching
    function of healing length given in the curly brackets.

 \vspace{0.15in} 
 \item   \textbackslash{}respa\_steps\_torsion\{0\} : \\
    Number of short time steps to be taken between updates of torsional
    forces.  A zero indicates torsional forces are updated every step.

 \vspace{0.15in} 
  \item   \textbackslash{}respa\_steps\_intra\{0\} : \\
    Number of short time steps to be taken between updates of 
    {\it intermolcular} forces.  A zero indicates intermolecular forces are 
    updated every step.

 \vspace{0.15in} 
 \item   \textbackslash{}shake\_tol\{1.0e-6\} : \\
     Constraints not treated by the group constraint method are iterated to a
     tolerance given in curly brackets.


 \vspace{0.15in} 
 \item   \textbackslash{}rattle\_tol\{1.0e-6\} : \\
      Time derivative of constraints not treated by the group constraint 
      method are iterated to a tolerance given in curly brackets.


 \vspace{0.15in} 
 \item   \textbackslash{}max\_constrnt\_iter\{200\} : \\
     Maximum number of iterations to be performed on constraints before a 
     warning that the tolerance has not yet been reached is printed out and 
     iteration stops.

 \vspace{0.15in} 
 \item   \textbackslash{}group\_con\_tol\{1.0e-6\} : \\
     Group constraint method is iterated to a tolerance given in curly 
     brackets.

\end{enumerate}

\newpage
%-------------------------------------------------------------------
\subsection*{\bf \~{}sim\_nhc\_def [28 keywords]}

\begin{enumerate}
 \vspace{0.15in} 
 \item   \textbackslash{}atm\_nhc\_tau\_def\{1000\} : \\
  Time scale (in femtoseconds) on which the thermostats should evolve. 
  Generally set to some characteristic time scale in the system.

 \vspace{0.15in} 
 \item   \textbackslash{}init\_resmp\_atm\_nhc\{on,{\bf off}\} : \\
   If turned on, atomic thermostat velocities will be initially resampled from
   a Maxwell distribution.

 \vspace{0.15in} 
 \item   \textbackslash{}atm\_nhc\_len\{2\} : \\
   Number of elements in the thermostat chain.  Generally 2-4 is adequate.

 \vspace{0.15in}
 \item   \textbackslash{}respa\_steps\_nhc\{2\} : \\
     Number of individual Suzuki/Yoshida factorizations to be employed in
     integration of the thermostat variables.  Only
     increase if energy conservation not satisfactory with default value.

 \vspace{0.15in}
 \item   \textbackslash{}yosh\_steps\_nhc\{1,{\bf 3},5,7\} : \\
   Order of the Suzuki/Yoshida integration scheme for the thermostat
   variables.  Only increase if energy conservation not satisfactory with
   default value.

 \vspace{0.15in} 
 \item   \textbackslash{}resmpl\_atm\_nhc\{0\} : \\
   Atomic thermostat velocities will be resampled with a frequency specified
   in the curly brackets.  A zero indicates no resampling of thermostat
   velocities.

 \vspace{0.15in}
 \item   \textbackslash{}respa\_xi\_opt\{{\bf 1},2,3,4\} : \\
   Depth of penetration of thermostat integration into multiple time step
   levels.  Larger numbers indicate deeper penetration.  A one indicates
   thermostat variables are updated at the beginning and end of 
   every step (XO option).

 \vspace{0.15in}
 \item   \textbackslash{}atm\_isokin\_opt\{on:on/off\}: \\
    Thermostat the atoms using isokinetic NHC (on) or plain NHC (off).

 \vspace{0.15in}
 \item   \textbackslash{}cp\_thermstats\{on:on/off\}: \\
    Thermostat the electrons (on) or run them without control (off).
    If ``on'', at least one isokinetic octopus NHC thermostat is placed on each
    g-space plane of each state.

 \vspace{0.15in}
 \item   \textbackslash{}cp\_num\_nhc\_iso\{2\}: \\
    The number of tenticles in an isokinetic octopus NHC thermostat.

 \vspace{0.15in}
 \item   \textbackslash{}cp\_nhc\_chunk\{1\}: \\
    The number of thermostats per g-space electronic state plane.
\end{enumerate}

\newpage
%-------------------------------------------------------------------
\subsection*{\bf \~{}sim\_write\_def [27 keywords]}
\begin{enumerate}

 \vspace{0.15in} 
 \item   \textbackslash{}sim\_name\{sim\_input.out\} : \\
    A complete list of all commands specified and defaults used.

 \vspace{0.15in} 
 \item   \textbackslash{}write\_screen\_freq\{1\} : \\
   Output information to screen will be written with a frequency specified
   in curly brackets.

 \vspace{0.15in} 
 \item   \textbackslash{}write\_dump\_freq\{1\} : \\
     The restart file of atomic coordinates and velocities, etc. will be
     written with a frequency specified in curly brackets.

 \vspace{0.15in}
 \item   \textbackslash{}write\_inst\_freq\{1\} : \\
     Instantaneous averages of energy, temperature, pressure, etc. will be 
     written to the instantaneous file with a frequency specified in curly 
     brackets.

 \vspace{0.15in} 
 \item   \textbackslash{}write\_pos\_freq\{100\} : \\
    Atomic positions will be appended to the trajectory position file with
    a frequency specified in curly brackets.

 \vspace{0.15in} 
 \item   \textbackslash{}write\_vel\_freq\{100\} : \\
    Atomic velocities will be appended to the trajectory velocity file with
    a frequency specified in curly brackets.

 \vspace{0.15in} 
 \item   \textbackslash{}sim\_name\{sim\_input.out\} : \\
     Name of file containing all keywords, both user set a default.

 \vspace{0.15in} 
 \item   \textbackslash{}out\_restart\_file\{sim\_restart.out\} : \\
   Name of the restart (dump) file containing final atomic 
          coordinates and velocities.

 \vspace{0.15in} 
 \item   \textbackslash{}in\_restart\_file\{sim\_restart.in\} : \\
     Name of the file containing initial atomic coordinates and
     velocities for this run.

 \vspace{0.15in} 
 \item   \textbackslash{}instant\_file\{sim\_instant.out\} : \\
     File to which instantaneous averages are to be written.

 \vspace{0.15in}
 \item   \textbackslash{}atm\_pos\_file\{sim\_atm\_pos.out\} : \\
     Trajectory position file name.

 \vspace{0.15in} 
 \item   \textbackslash{}atm\_vel\_file\{sim\_atm\_vel.out\} : \\
  Trajectory force file name.  

 \vspace{0.15in} 
 \item   \textbackslash{}atm\_force\_file\{sim\_atm\_force.out\} : \\
     Trajectory velocity file name.

 \vspace{0.15in}
 \item   \textbackslash{}conf\_partial\_file\{sim\_atm\_pos\_part.out\} : \\
     Partial trajectory position file name.

 \vspace{0.15in} 
 \item   \textbackslash{}mol\_set\_file\{sim\_atm\_mol\_set.in\} : \\
    Name of file containing instructions on how to name and build molecules.
    Detailed instructions on contents of this file given elsewhere.

 \vspace{0.15in} 
 \item   \textbackslash{}write\_force\_freq\{1000000\} : \\
   Atomic forces are written to the trajectory force file with a
   frequency specified in the curly brackets.

 \vspace{0.15in} 
 \item   \textbackslash{}screen\_output\_units\{{\bf au},kcal\_mol,kelvin\} : \\
   Output information is written to the screen in units specified 
   in curly brackets.

 \vspace{0.15in} 
 \item   \textbackslash{}conf\_file\_format\{{\bf binary},formatted\} : \\
   Trajectory files are written either in binary or formatted form.

 \vspace{0.15in} 
 \item   \textbackslash{}conf\_partial\_limits\{1,0\} : \\
   Two numbers specifying the first and last atoms to be written to
   the partial trajectory position file.

\end{enumerate}

\newpage
%-------------------------------------------------------------------
\subsection*{\bf \~{}sim\_list\_def [14 keywords]}

\begin{enumerate}

 \vspace{0.15in} 
 \item   \textbackslash{}neighbor\_list\{no\_list,{\bf ver\_list},lnk\_list\} : \\
   Type of neighbor list to use.  Optimal scheme for most systems: 
   \textbackslash{}ver\_list with a \textbackslash{}lnk\_lst update type.

 \vspace{0.15in} 
 \item   \textbackslash{}update\_type\{{\bf lnk\_lst},no\_list\} : \\
   Update the verlet list using either no list or a link list.

 \vspace{0.15in} 
 \item   \textbackslash{}verlist\_skin\{1\} : \\
   Skin depth added to verlet list cutoff.  Needs to be optimized between
   extra neighbors added and average required update frequency.

 \vspace{0.15in} 
 \item   \textbackslash{}lnk\_cell\_divs\{7\} : \\
   Number of link cell divisions in each dimension.  Needs to be optimized
   if used either as list type or update type.

\end{enumerate}

\newpage
%-------------------------------------------------------------------
\subsection*{\bf \~{}sim\_class\_PE\_def [25 keywords]}

\begin{enumerate}

 \vspace{0.15in} 
 \item   \textbackslash{}ewald\_kmax\{7\} : \\
  Maximum length of k-vectors used in evaluating Ewald summation for 
  electrostatic interactions.

 \vspace{0.15in} 
 \item   \textbackslash{}ewald\_alpha\{7\} : \\
  Size of real-space damping (screening) parameter in Ewald sum.

 \vspace{0.15in} 
 \item   \textbackslash{}ewald\_respa\_kmax\{0\} : \\
   Maximum length of k-vectors used in long range force RESPA integration.

 \vspace{0.15in}
 \item   \textbackslash{}ewald\_pme\_opt\{on,{\bf off}\} : \\
  If turned on, then electrostatic interactions are evaluated using the
  smooth particle mesh method.  Its use is recommended.

 \vspace{0.15in} 
 \item   \textbackslash{}ewald\_kmax\_pme\{7\} : \\
  Maximum length of k-vectors used in evaluating Ewald summation for 
  electrostatic interactions via the smooth particle mesh method.

 \vspace{0.15in} 
 \item   \textbackslash{}ewald\_interp\_pme\{4\} : \\
   Order of mesh interpolation in the smooth particle mesh Ewald method.

 \vspace{0.15in} 
 \item   \textbackslash{}ewald\_respa\_pme\_opt\{on,{\bf off}\} : \\
  If turned on, then long range force RESPA is used in conjunction with the
  smooth particle mesh method.

 \vspace{0.15in} 
 \item   \textbackslash{}ewald\_respa\_interp\_pme\{4\} : \\
  Order of mesh interpolation in the smooth particle mesh Ewald method used
  with RESPA.

 \vspace{0.15in}
 \item   \textbackslash{}sep\_VanderWaals\{on,{\bf off}\} : \\
   If turned on, then the VanderWaals component of the {\it inter}molecular 
   interaction energy is printed separately to screen.

 \vspace{0.15in} 
 \item   \textbackslash{}inter\_spline\_pts\{2000\} : \\
    Number of spline points used {\it inter}molecular interaction 
    potential and derivative.

\end{enumerate}

\newpage
%-------------------------------------------------------------------
\subsection*{\bf \~{}sim\_vol\_def [4 keywords]}

\begin{enumerate}

 \vspace{0.15in} 
 \item  \textbackslash{}volume\_tau\{1000\} : \\
     Time scale of volume evolution
     under constant pressure (NPT\_I or NPT\_F) in femtoseconds.
  
 \vspace{0.15in} 
 \item  \textbackslash{}volume\_nhc\_tau\{1000\} : \\
     Time scale of volume 
     heat bath evolution under constant pressure in femtoseconds.

\end{enumerate}

\newpage
%-------------------------------------------------------------------
\subsection*{\bf \~{}sim\_cp\_def [29 keywords]}

\begin{enumerate}

 \vspace{0.15in} 
 \item  \textbackslash{}cp\_e\_e\_interact\{{\bf on},off\} : \\
     Electron-electron interactions can be turned off for 1e problems.

 \vspace{0.15in} 
 \item  \textbackslash{}cp\_dft\_typ\{{\bf lda},lsda,gga\_lda,gga\_lsda\} : \\
     Density function type where gga mean gradient corrections are on.

 \vspace{0.15in} 
 \item  \textbackslash{}cp\_vxc\_typ\{{\bf pz\_lda},pw\_lda,pz\_lsda\} : \\
     Exchange correlation function type.

 \vspace{0.15in} 
 \item  \textbackslash{}cp\_ggax\_typ\{becke,pw91x,fila\_1x,fil\_2x,{\bf off}\} : \\
     Gradient corrected exchange function type.

 \vspace{0.15in} 
 \item  \textbackslash{}cp\_ggac\_typ\{lyp,lypm1,pw91c,{\bf off}\} : \\
     Gradient corrected correlation function type.

 \vspace{0.15in} 
 \item  \textbackslash{}cp\_norb\{full\_ortho,norm\_only,{\bf off}\} : \\
     Non-orthogonal orbitals option (see Tuckerman-Hutter-Parr. papers).

 \vspace{0.15in} 
 \item  \textbackslash{}cp\_minimize\_typ\{{\bf min\_std},min\_cg\} : \\
      Minimization type, steepest descent, conjugate gradient 

 \vspace{0.15in} 
 \item  \textbackslash{}cp\_mass\_tau\_def\{25\} : \\
      CP fictitious dynamics time scale in femtoseconds.

 \vspace{0.15in} 
 \item  \textbackslash{}cp\_energy\_cut\_def\{2\} : \\
      Plane wave expansion cutoff $E_g<=\hbar^2g^2/2m_e$ 

 \vspace{0.15in} 
 \item  \textbackslash{}cp\_mass\_cut\_def\{2\} : \\
      Renormalize CP masses for $E_g<=\hbar^2g^2/2m_e$ Rhydberg.

 \vspace{0.15in} 
 \item  \textbackslash{}cp\_fict\_KE\{1000\} : \\
      Energy give to the plane waves to create the Born-Oppenheimer
      surface with there fake dynamics

 \vspace{0.15in} 
 \item  \textbackslash{}cp\_ptens\{on,{\bf off}\} : \\
      Evaluate the pressure tensor. 

 \vspace{0.15in} 
 \item  \textbackslash{}cp\_init\_orthog\{on,{\bf off}\} : \\
      Orthogonalize the states at the start.

 \vspace{0.15in} 
 \item  \textbackslash{}cp\_orth\_meth\{{\bf gram\_schmidt},lowdin\} : \\
      Orthogonalize using gram-schmidt or lowdin.

 \vspace{0.15in} 
 \item  \textbackslash{}cp\_restart\_type\{{\bf gen\_wave},initial,
                         restart\_pos,restart\_posvel,restart\_all\} : \\
      Start the run by generating sum guess to the states, ...

 \vspace{0.15in} 
 \item  \textbackslash{}cp\_nonloc\_ees\_opt\{on:on/off \}: \\     
 Compute the non-local pseudopotential energy and forces using Euler
 exponential spline interpolation (EES).

 \vspace{0.15in} 
 \item  \textbackslash{}cp\_eext\_ees\_opt\{on:on/off \}: \\     
     Compute the local pseudopotential energy and forces as well as the 
     reciprocal space part of the Ewald sum using  EES.

 \vspace{0.15in} 
 \item  \textbackslash{}cp\_pseudo\_ees\_order\{6 \}: \\     
     Use a EES interpolation order of 6.

 \vspace{0.15in} 
 \item  \textbackslash{}cp\_pseudo\_ees\_scale\{1.4 \}: \\     
     Use a g-space expanded 1.4x to perform the EES interpolation.
\end{enumerate}

\newpage
%-------------------------------------------------------------------
\subsection*{\bf \~{}sim\_pimd\_def [10 keywords]}

\begin{enumerate}

 \vspace{0.15in} 
 \item  \textbackslash{}path\_int\_beads\{1\} : \\
     Number of path integral beads.

 \vspace{0.15in} 
 \item  \textbackslash{}path\_int\_md\_typ\{staging,centroid\} : \\
     Path integral molecular dynamics type, staging or centroid/normal mode.
     Staging gives the fastest equilibration.

 \vspace{0.15in} 
 \item  \textbackslash{}path\_int\_gamma\_adb\{1\} : \\
     Path integral molecular dynamics type adiabaticity parameter.
     Employed with the centroid option to give approx. to quantum dynamics.

 \vspace{0.15in} 
 \item  \textbackslash{}respa\_steps\_pimd\{1\} : \\
     Path integral bead respa steps.

 \vspace{0.15in} 
 \item  \textbackslash{}initial\_spread\_size\{1\} : \\
    In growing cyclic initial paths, what is the how large do you
    want their diameter or spread in Angstroms.

 \vspace{0.15in} 
 \item  \textbackslash{}initial\_spread\_opt\{on,{\bf off}\} : \\
    Do you want initial paths grown for you?

\end{enumerate}

\newpage
%----------------------------------------------------------------
\newpage
%-------------------------------------------------------------------
\subsection*{\bf \~{}sim\_temper\_def [10 keywords]}

\begin{enumerate}

 \vspace{0.15in} 
 \item  \textbackslash{}num\{1\} : \\
   Number of temperers

 \vspace{0.15in} 
 \item  \textbackslash{}ens\_opt\{nvt\} : \\
   Ensemble in which you are tempering

 \vspace{0.15in} 
 \item  \textbackslash{}statept\_infile\{statepoint.list\} : \\
   List of statepoints to be tempered.

 \vspace{0.15in} 
 \item  \textbackslash{}screen\_name\{screen.out\} : \\
   Output name of any screen output for each temperer will be screen.out.\#

 \vspace{0.15in} 
 \item  \textbackslash{}switch\_steps\{1000\} : \\
   Number of steps between switches

 \vspace{0.15in} 
 \item  \textbackslash{}output\_directory\{TEMPER\_OUT\} : \\
   Output directory where files are stored.

\end{enumerate}
\newpage
%-------------------------------------------------------------------
%==================================================================



%==================================================================
%==================================================================
\section{\bf Molecule and Wave function Keyword Dictionary} 

The following commands maybe specified in the system setup file:
\begin{enumerate}
\item {\bf \~{}molecule\_def []}
\item {\bf \~{}wavefunc\_def []}
\item {\bf \~{}bond\_free\_def []}
\item {\bf \~{}data\_base\_def []}
\end{enumerate}

%==================================================================

\newpage
%-------------------------------------------------------------------
\subsection*{\bf \~{}molecule\_def}

\begin{enumerate}

 \vspace{0.15in} 
 \item  \textbackslash{}mol\_name\{\} : \\
   The name of the molecule type.

 \vspace{0.15in} 
 \item  \textbackslash{}num\_mol\{ \} : \\ 
   The number of this molecule type you want in your system.

 \vspace{0.15in} 
 \item  \textbackslash{}num\_residue\{ \} : \\
   The number of residues in in this molecule type.

 \vspace{0.15in} 
 \item  \textbackslash{}mol\_index\{2\} : \\ 
   This is the ``2nd'' molecule type defined in the system.

 \vspace{0.15in} 
 \item  \textbackslash{}mol\_parm\_file\{ \} : \\
   The topology of the molecule type is described in this file.

 \vspace{0.15in} 
 \item  \textbackslash{}mol\_text\_nhc\{ \} : \\
   The temperature of this molecule type. It may be different than the
   external temperature for fancy simulation studies in the adiabatic limit.

 \vspace{0.15in} 
 \item  \textbackslash{}mol\_freeze\_opt\{{\bf none},all,backbone\} : \\
    Freeze this molecule to equilibrate the system.

 \vspace{0.15in} 
 \item  \textbackslash{}hydrog\_mass\_opt\{A,B\} : \\
   Increase the mass of type A hydrogens where A is 
   {\bf off},all,backbone or sidechain to B amu. This allows
   selective deuteration for example.

 \vspace{0.15in} 
 \item  \textbackslash{}hydrog\_con\_opt\{{\bf off},all,polar\} : \\
   Constrain all bonds to this type of hydrogen in the molecule.

 \vspace{0.15in} 
 \item  \textbackslash{}mol\_nhc\_opt\{none,{\bf global},glob\_mol,ind\_mol,\\ 
                 \hspace*{0.5in}res\_mol,atm\_mol,mass\_mol\} :  \\
   Nose-Hoover Chain option: Couple the atoms in this molecule to:
      \begin{itemize}
         \item none:     no thermostats.
         \item global:   the global thermostat.
         \item glob\_mol: a thermostat for this molecule type. 
         \item ind\_mol:  a thermostat for each molecule of this type. 
         \item res\_mol:  a thermostat for each residue in each molecule.
         \item atm\_mol:  a thermostat for each atom in each molecule.
         \item mass\_mol: a thermostat for each degree of freedom.
      \end{itemize}

 \vspace{0.15in} 
 \item  \textbackslash{}mol\_tau\_nhc\{ \} : \\
   Nose-Hoover Chain time scale in femtoseconds for this molecule type.

\end{enumerate}

\newpage
%-------------------------------------------------------------------
\subsection*{\bf \~{}wavefunc\_def}

\begin{enumerate}

 \vspace{0.15in} 
 \item  \textbackslash{}nstate\_up\{1\} : \\ 
     Number of spin up states in the spin up electron density.

 \vspace{0.15in} 
 \item  \textbackslash{}nstate\_dn\{1\} : \\
     Number of spin down states in the spin down electron density.

 \vspace{0.15in} 
 \item  \textbackslash{}cp\_tau\_nhc\{25\} : \\
     Nose-Hoover Chain coupling time scale.

 \vspace{0.15in} 
 \item  \textbackslash{}num\_kpoint\{1\} : \\
  Number of kpoints. If it is not set (default), code assumes gamma point.

 \vspace{0.15in} 
 \item  \textbackslash{}kpoint\_file\{pi\_md.kpts\} : \\
 File where list of kpoints is stored. First line is number of kpoints followed
by list of kpoints and weights each on a separate line (ka kb kc  wght) for a 
file of nkpoint + 1 lines.

\end{enumerate}

\newpage
%-------------------------------------------------------------------
\subsection*{\bf \~{}bond\_free\_def}

\begin{enumerate}

 \vspace{0.15in} 
 \item  \textbackslash{}atom1\_moltyp\_ind\{ \} : \\ 
    Index of molecule type to which atom 1 belongs.

 \vspace{0.15in} 
 \item  \textbackslash{}atom2\_moltyp\_ind\{ \} : \\ 
    Index of molecule type to which atom 2 belongs.

 \vspace{0.15in} 
 \item  \textbackslash{}atom1\_mol\_ind\{ \} : \\    
    Index of the molecule of the specified molecule type 
    to which atom 1 belongs.

 \vspace{0.15in} 
 \item  \textbackslash{}atom2\_mol\_ind\{ \} : \\    
    Index of the molecule of the specified molecule type 
    to which atom 2 belongs.

 \vspace{0.15in} 
 \item  \textbackslash{}atom1\_residue\_ind\{ \} : \\ 
    Index of the residue in the molecule to which atom 1 belongs.

 \vspace{0.15in} 
 \item  \textbackslash{}atom2\_residue\_ind\{ \} : \\ 
    Index of the residue in the molecule to which atom 2 belongs.

 \vspace{0.15in} 
 \item  \textbackslash{}atom1\_atm\_ind\{ \} : \\    
    Index of the atom 1 in the residue.
 
 \vspace{0.15in} 
 \item  \textbackslash{}atom2\_atm\_ind\{ \} : \\     
    Index of the atom 2 in the residue.
 
 \vspace{0.15in} 
 \item  \textbackslash{}eq\{ \} : \\               
    Equilibrium bond length in Angstrom.
  
 \vspace{0.15in} 
 \item  \textbackslash{}fk\{ \} : \\               
    Umbrella sampling force constant in K/Angstro${\rm m}^2$.

 \vspace{0.15in} 
 \item  \textbackslash{}rmin\_hist\{ \} : \\         
    Min Histogram distance in Angstrom.

 \vspace{0.15in} 
 \item  \textbackslash{}rmax\_hist\{ \} : \\         
    Max Histogram distance in Angstrom.

 \vspace{0.15in} 
 \item  \textbackslash{}num\_hist\{ \} : \\         
    Number of points in the histogram.
 
 \vspace{0.15in} 
 \item  \textbackslash{}hist\_file\{ \} : \\        
    File to which the histogram is printed.

\end{enumerate}

\newpage
%-------------------------------------------------------------------
\subsection*{\bf \~{}data\_base\_def}

\begin{enumerate}

 \vspace{0.15in} 
 \item  \textbackslash{}inter\_file\{ \} : \\
    File containing intermolecular interaction parameters.

 \vspace{0.15in} 
 \item  \textbackslash{}vps\_file\{pi\_md.inter\} : \\
    File containing pseudopotential interaction parameters.

 \vspace{0.15in} 
 \item  \textbackslash{}bond\_file\{pi\_md.bond\} : \\
    File containing bond interaction parameters.

 \vspace{0.15in} 
 \item  \textbackslash{}bend\_file\{pi\_md.bend\} : \\
    File containing bend interaction parameters.

 \vspace{0.15in} 
 \item  \textbackslash{}tors\_file\{pi\_md.tors\} : \\
    File containing torsion interaction parameters.

 \vspace{0.15in} 
 \item  \textbackslash{}onefour\_file\{pi\_md.onfo\} : \\
    File containing onefour interaction parameters.

\end{enumerate}

%-------------------------------------------------------------------
\newpage
%==================================================================


%==================================================================
%==================================================================
\section{\bf Topology Keyword Dictionary} 

The following commands maybe specified in the molecular topology 
and parameter files:
\begin{enumerate}
\item {\bf \~{ }molecule\_name\_def []}
\item {\bf \~{ }residue\_def []}
\item {\bf \~{ }residue\_bond\_def []}
\item {\bf \~{ }residue\_name\_def []}
\item {\bf \~{ }atom\_def []} 
\item {\bf \~{ }bond\_def []}
\item {\bf \~{ }grp\_bond\_def []}
\item {\bf \~{ }bend\_def []}
\item {\bf \~{ }bend\_bnd\_def []}
\item {\bf \~{ }tors\_def []}
\item {\bf \~{ }onfo\_def   []}   
\item {\bf \~{ }residue\_morph []}
\item {\bf \~{ }atom\_destroy []}
\item {\bf \~{ }atom\_create []}
\item {\bf \~{ }atom\_morph []}
\end{enumerate}

%==================================================================
\newpage
%-------------------------------------------------------------------
\subsection*{\bf \~{ }molecule\_name\_def []}

\begin{enumerate}

 \vspace{0.15in} 
 \item  \textbackslash{}molecule\_name\{\} : \\ 
    Molecule name in characters.

 \vspace{0.15in} 
 \item  \textbackslash{}nresidue\{\} : \\ 
    Number of residues in the molecule.

 \vspace{0.15in} 
 \item  \textbackslash{}natom\{\} : \\ 
    Number of atoms in the molecule.
\end{enumerate}

\newpage
%-------------------------------------------------------------------
\subsection*{\bf \~{ }residue\_def []}

\begin{enumerate}

 \vspace{0.15in} 
 \item  \textbackslash{}residue\_name\{\} : \\ 
    Residue name in characters.

 \vspace{0.15in} 
 \item  \textbackslash{}residue\_index\{\} : \\ 
    Residue index number (this is the third residue in the molecule).

 \vspace{0.15in} 
 \item  \textbackslash{}natom\{\} : \\ 
    Number of atoms in the residue (after morphing).

 \vspace{0.15in} 
 \item  \textbackslash{}residue\_parm\_file\{\} : \\ 
    Residue parameter file.

 \vspace{0.15in} 
 \item  \textbackslash{}residue\_fix\_file\{\} : \\ 
    File containing morphing instructions for this residue if any.

\end{enumerate}

\newpage
%-------------------------------------------------------------------
\subsection*{\bf \~{ }residue\_bond\_def []}

\begin{enumerate}

 \vspace{0.15in} 
 \item  \textbackslash{}res1\_typ\{\} : \\ 
    Residue name in characters.

 \vspace{0.15in} 
 \item  \textbackslash{}res2\_typ\{\} : \\ 
    Residue name in characters.

 \vspace{0.15in} 
 \item  \textbackslash{}res1\_index\{\} : \\ 
    Numerical index of first residue.

 \vspace{0.15in} 
 \item  \textbackslash{}res2\_index\{\} : \\ 
    Numerical index of second residue.

 \vspace{0.15in} 
 \item  \textbackslash{}res1\_bond\_site\{\} : \\ 
   Connection point of residue bond in residue 1.

 \vspace{0.15in} 
 \item  \textbackslash{}res2\_bond\_site\{\} : \\ 
   Connection point of residue bond in residue 2.

 \vspace{0.15in} 
 \item  \textbackslash{}res1\_bondfile\{\} : \\ 
   Morph file for residue 1 (lose a hydrogen, for example).

 \vspace{0.15in} 
 \item  \textbackslash{}res2\_bondfile\{\} : \\ 
   Morph file for residue 2 (lose an halogen atom, for example).
  
\end{enumerate}

\newpage
%-------------------------------------------------------------------
\subsection*{\bf \~{ }residue\_name\_def []}

\begin{enumerate}

 \vspace{0.15in} 
 \item  \textbackslash{}res\_name\{\} : \\ 
   Name of the residue in characters.

 \vspace{0.15in} 
 \item  \textbackslash{}natom\{\} : \\ 
   Number of atoms in the residue
\end{enumerate}

\newpage
%-------------------------------------------------------------------
\subsection*{\bf \~{ }atom\_def []} 

\begin{enumerate}

 \vspace{0.15in} 
 \item  \textbackslash{}atom\_typ\{\} : \\ 
   Type of atom in characters.
 
 \vspace{0.15in} 
 \item  \textbackslash{}atom\_ind\{\} : \\ 
   This is the nth atom in the molecule. The number is used to 
   define bonds, bends, tors, etc involving this atom.

 \vspace{0.15in} 
 \item  \textbackslash{}mass\{\} : \\ 
    The mass of the atom in amu.

 \vspace{0.15in} 
 \item  \textbackslash{}charge\{\} : \\ 
    The charge on the atom in ``e''.

 \vspace{0.15in} 
 \item  \textbackslash{}valence\{\} : \\ 
    The valence of the atom.

 \vspace{0.15in} 
 \item  \textbackslash{}improper\_def\{0,1,2,3\} : \\ 
    How does this atom like its improper torsion constructed (if any).

 \vspace{0.15in} 
 \item  \textbackslash{}bond\_site\_1\{site,prim-branch,sec-branch\} : \\ 
    To what bond-site(s), character string, does this atom belong and where is
    it in the topology tree structure relative to the root (the atom
    that will bonded to some incoming atom).
      \begin{itemize}
        \item Atom to be bonded =root atom  (0,0).
        \item 1st atom bonded to root atom  (1,0).
        \item 2nd atom bonded to root atom  (2,0).
        \item 1st atom bonded to 1st branch (1,1).
        \item 2nd atom bonded to 1st branch (1,2).
      \end{itemize}


 \vspace{0.15in} 
 \item  \textbackslash{}def\_ghost1\{index,coeff\} : \\ 
    If the atom is a ghost (these funky sites that TIP4P type
    models have), it spatial position is constructed by taking linear 
    combinations of the other atoms in the molecule specified by the index 
    and the coef. A ghost can be composed of up to six other atoms
    (ghost\_1{} ... ghost\_6{}).

\end{enumerate}

\newpage
%-------------------------------------------------------------------
\subsection*{\bf \~{ }bond\_def []}

\begin{enumerate}
 \vspace{0.15in} 
 \item  \textbackslash{}atom1\{\} : \\
   Numerical index of atom 1.

 \vspace{0.15in} 
 \item  \textbackslash{}atom2\{\} : \\
   Numerical index of atom 2.

 \vspace{0.15in} 
 \item  \textbackslash{}modifier\{con,{\bf on},off\} : \\
   The bond is active=on, inactive=off, or constrained=con.

\end{enumerate}

\newpage
%-------------------------------------------------------------------

\subsection*{\bf \~{ }residue\_morph []}
\begin{itemize}
  \item Used to define a file which contains modifications to 
              a given residue.
  \item It has the same arguments as \~{ }residue\_name\_def[].
\end{itemize}

%-------------------------------------------------------------------

\subsection*{\bf \~{ }atom\_destroy []}
\begin{itemize}
   \item Used to destroy an atom in a residue morph file.
   \item Uses the same key words as \~{ }atom\_def[]
\end{itemize}

%-------------------------------------------------------------------

\subsection*{\bf \~{ }atom\_create []}
\begin{itemize}
   \item Used to create an atom in a residue morph file.
   \item Uses the same key words as \~{ }atom\_def[]
\end{itemize}

%-------------------------------------------------------------------

\subsection*{\bf \~{ }atom\_morph []}
\begin{itemize}
   \item Used to morph an atom in a residue morph file.
   \item Uses the same key words as \~{ }atom\_def[]
\end{itemize}

%-------------------------------------------------------------------
\newpage
%==================================================================


%==================================================================
%==================================================================
\section{\bf Potential Parameter Keyword Dictionary} 

The following commands maybe specified in the potential parameter files:
\begin{enumerate}
\item {\bf \~{ }inter\_parm []}
\item {\bf \~{ }bond\_parm []}
\item {\bf \~{ }bend\_parm []}
\item {\bf \~{ }tors\_parm []}
\item {\bf \~{ }onfo\_parm []}
\item {\bf \~{ }pseudo\_parm []}
\item {\bf \~{ }bend\_bnd\_parm []}
\end{enumerate}


%==================================================================

\newpage
%-------------------------------------------------------------------
\subsection*{\bf \~{}inter\_parm}

\begin{enumerate}

 \vspace{0.15in} 
 \item  \textbackslash{}atom1\{\} : \\ 
    Atom type 1 (character data).

 \vspace{0.15in} 
 \item  \textbackslash{}atom2\{\} : \\ 
    Atom type 2 (character data).

 \vspace{0.15in} 
 \item  \textbackslash{}pot\_type\{lennard-jones,williams,aziz-chen,null\} : \\ 
    Potential type: Williams is an exponential c6-c8-c10. Aziz-chen has
    a short range switching function on the vanderwaals part.

 \vspace{0.15in} 
 \item  \textbackslash{}min\_dist\{\} : \\ 
    Minimum interaction distance.

 \vspace{0.15in} 
 \item  \textbackslash{}max\_dist\{\} : \\ 
    Spherical cutoff interaction distance.

 \vspace{0.15in} 
 \item  \textbackslash{}res\_dist\{\} : \\ 
    Respa Spherical cutoff interaction distance.

 \vspace{0.15in} 
 \item  \textbackslash{}sig\{\} : \\ 
    Lennard-Jones parameter.

 \vspace{0.15in} 
 \item  \textbackslash{}eps\{\} : \\ 
    Lennard-Jones parameter.

 \vspace{0.15in} 
 \item  \textbackslash{}c6\{\} : \\ 
    Williams/Aziz-Chen Vdw parameter.

 \vspace{0.15in} 
 \item  \textbackslash{}c8\{\} : \\ 
    Williams/Aziz-Chen Vdw parameter.

 \vspace{0.15in} 
 \item  \textbackslash{}c9\{\} : \\ 
    Williams/Aziz-Chen Vdw parameter.

 \vspace{0.15in} 
 \item  \textbackslash{}c10\{\} : \\ 
    Williams/Aziz-Chen Vdw parameter.

 \vspace{0.15in} 
 \item  \textbackslash{}Awill\{\} : \\ 
    Williams/Aziz-Chen parameter ($A\exp(-Br-Cr^2)$).

 \vspace{0.15in} 
 \item  \textbackslash{}Bwill\{\} : \\ 
    Williams/Aziz-Chen parameter ($A\exp(-Br-Cr^2)$).

 \vspace{0.15in} 
 \item  \textbackslash{}Cwill\{\} : \\ 
    Williams/Aziz-Chen parameter ($A\exp(-Br-Cr^2)$).

 \vspace{0.15in} 
 \item  \textbackslash{}rm\_swit\{\} : \\ 
    Williams/Aziz-Chen parameter controlling Vdw switching 
   ($f(r,r_m) = $).

\end{enumerate}

\newpage
%-------------------------------------------------------------------
\subsection*{\bf \~{}bond\_parm}

\begin{enumerate}

 \vspace{0.15in} 
 \item  \textbackslash{}atom1\{\} : \\
    Atom type 1 (character data).
   
 \vspace{0.15in} 
 \item  \textbackslash{}atom2\{\} : \\ 
    Atom type 2 (character data).

 \vspace{0.15in} 
 \item  \textbackslash{}pot\_type\{{\bf harmonic},power-series,morse,null\} : \\
     Potential type of bond.

 \vspace{0.15in} 
 \item  \textbackslash{}fk\{\} : \\
     Force constant of bond in K/Angstro${\rm m}^2$.

 \vspace{0.15in} 
 \item  \textbackslash{}eq\{\} : \\
    Equilibrium bond length in Angstrom.

 \vspace{0.15in} 
 \item  \textbackslash{}eq\_res\{\} : \\
     Respa Equilibrium bond length in Angstrom.

 \vspace{0.15in} 
 \item  \textbackslash{}alpha\{\} : \\
    Morse parameter alpha in inverse Angstrom \\
    ($\phi(r) = d_0[\exp(-\alpha(r-r_0))-1]^2$).

 \vspace{0.15in} 
 \item  \textbackslash{}d0\{\} : \\
    Morse d0 in Kelvin
    ($\phi(r) = d_0[\exp(-\alpha(r-r_0))-1]^2$).

\end{enumerate}

\newpage
%-------------------------------------------------------------------
\subsection*{\bf \~{}pseudo\_parm}

\begin{enumerate}

 \vspace{0.15in} 
 \item  \textbackslash{}atom1\{\} : \\
    Atom type in character data.

 \vspace{0.15in} 
 \item  \textbackslash{}vps\_typ\{local,kb,vdb,null\} : \\
    Pseudopotential type: local, Kleinman-Bylander, Vanderbilt or null.

 \vspace{0.15in} 
 \item  \textbackslash{}vps\_file\{\} : \\
    File containing the numerical generated pseudopotentials.

 \vspace{0.15in} 
 \item  \textbackslash{}n\_ang\{\} : \\
    Number of angular momentum projection operators required.

 \vspace{0.15in} 
 \item  \textbackslash{}loc\_opt\{\} : \\
    Which angular momentum pseudopotential is taken to be the local.

\end{enumerate}

%------------------------------------------------------------------
\newpage
%==================================================================

%==================================================================
%==================================================================
\section{\bf Coordinate Input Files} 

At present the code will build topologies for the user. However,
it will not build coordinates. Therefore, the user must provide
an initial input coordinate file( see \textbackslash{}restart\_type\{initial\}
and \textbackslash{}in\_restart\_file\{sim\_restart.in\})
The first line of the restart file must contain the number of atoms,
followed by a one, followed by the number of path integral beads 
(usually 1). The code will grow beads and only P=1 need be provided
initially. The atoms positions then follow in the order specified by
the set file and the topology files. Finally, at the bottom of
the code the 3x3 simulation cell matrix must be specified.
The unit is Angstrom.

Example: \\ 
The molecular set file specifies 25 water molecules as molecule
type 1 and 3 bromine atom as molecule type 2. The water parameter file is 
written as OHH, i.e. oxygen is atom 1, hydrogen1 is atom 2, hydrogen2 is 
atom 3. Therefore, the input file should have 25 OHH coordinates followed
3 bromines. If the box is a square 25 angstrom on edge 
then the last three lines of the file should be \\
\hspace*{1.5in}  25  0  0  \\
\hspace*{1.5in}  0  25  0  \\
\hspace*{1.5in}  0  0   25 \\

\newpage
%==================================================================
%==================================================================
\section{\bf Parallel Decomposition Keyword Dictionary} 

The following commands maybe specified in the parallel decomposition file:
\begin{enumerate}
\item {\bf \~{ }charm\_conf\_gen\_def[ ]}
\item {\bf \~{ }charm\_conf\_rho\_def[ ]}
\item {\bf \~{ }charm\_conf\_state\_def[ ]}
\item {\bf \~{ }charm\_conf\_PC\_def[ ]}
\item {\bf \~{ }charm\_conf\_NL\_def[ ]}
\item {\bf \~{ }charm\_conf\_map\_def[ ]}
\end{enumerate}
If no commands are set, the charm++ driver will decide values based
on system size and processor number and architecture. Below, the 
name nstate refers to the number of KS states and the name nproc
refers to the number of processors to be used.

%==================================================================
\newpage
%-------------------------------------------------------------------
\subsection*{\bf \~{ }charm\_conf\_gen\_def}

\begin{enumerate}
  \vspace{0.15in} 
  \item  \textbackslash{}useCommlib\{on:on,off \} : \\
  Turn on/off Charm communication library use. Default on
  \vspace{0.15in} 
  \item  \textbackslash{}useCommlibMulticast\{ \} : \\   
  \vspace{0.15in} 
  Turn on/off Charm multicast communication library use. Default on
  \item  \textbackslash{}atmOutput\{on:on,off \} : \\   
  Turn on/off atom output. Default on
\end{enumerate}

\newpage
%-------------------------------------------------------------------
\subsection*{\bf \~{ }charm\_conf\_rho\_def}

\begin{enumerate}
  \vspace{0.15in} 
  \item \textbackslash{}gExpandFactRho\{1 \} : \\    
  Expands the g-space parallelization of the electron density beyond
  the number of planes by the specified non-integer multiplicative factor. 
  \vspace{0.15in} 
  \item \textbackslash{}rhoGHelpers\{1 \} : \\    
  Expands the g-space parallelization of the local pseudopotential
  evaluation by the integer increment.
  \vspace{0.15in} 
  \item \textbackslash{}rhoRsubplanes\{1 \} : \\    
  Expands the r-space parallelization of the exchange correlation and 
  local pseudopotential energies by the integer increment. 
  The latter only matters if EES is on.
  \vspace{0.15in} 
  \item \textbackslash{}rhoLineOrder\{skip:skip,none,random \} : \\    
  Static Load balance scheme for density g-space parallelization.
  \vspace{0.15in} 
  \item \textbackslash{}nchareHartAtmT\{1 : $1 \leq$ $\leq$ natmTyp\} : \\ 
  Decomposition for the local pseudopotentil energy for use with the 
  EES method.
  \vspace{0.15in} 
  \item \textbackslash{}rhoSubPlaneBalance\{off:on/off \} : \\    
  When rhoRsubplanes$>1$, balance the communication.
  \vspace{0.15in} 
  \item \textbackslash{}rhoGToRhoRMsgCombine\{off:on/off \} : \\    
  When rhoRsubplanes$>1$, combine messages.
\end{enumerate}

\newpage
%-------------------------------------------------------------------
\subsection*{\bf \~{ }charm\_conf\_state\_def}

\begin{enumerate}
  \vspace{0.15in} 
  \item \textbackslash{}dataPath\{./STATES \} : \\    
  The directory in which the electronic state input files are
  stored. If the ``gen\_wave'' option is selected, the dataPath
  command is ignored.
  \vspace{0.15in} 
  \item \textbackslash{}dataPathOut\{./STATES\_OUT \} : \\    
  The directory in which the electronic state output files are stored.
  \vspace{0.15in} 
  \item \textbackslash{}gExpandFact\{1.0 \} : \\    
  Expands the g-space parallelization of the electronic states beyond the number of planes by
  the specified non-integer multiplicative factor. If the torus map option is
  employed, the product must be a power of 2, for example 
 (gexpandfact=2)x(nplane\_state=4)=8.
  \vspace{0.15in} 
  \item \textbackslash{}stateOutput\{on: on/off \} : \\    
  Turn the electronic state input on and off.
\end{enumerate}

\newpage
%-------------------------------------------------------------------
\subsection*{\bf \~{ }charm\_conf\_PC\_def}

\begin{enumerate}
  \vspace{0.15in} 
  \item \textbackslash{}sGrainSize\{ nstate\} : \\    
  Integer smatrix decomposition parameter. The default value is number of
  electron states, e.g. the matrix is not decomposed.
  \vspace{0.15in} 
  \item \textbackslash{}orthoGrainSize\{ nstate\} : \\    
  Integer lowdix decomposition parameter. The default value is number of
  electron states, e.g. the matrix is not decomposed. This parameter
  must be $\leq$ sGrainSize and ${\rm mod}(sGrainSize,oGrainSize)=0$.
  \vspace{0.15in} 
  \item \textbackslash{}numChunks\{ 1 \} : \\    
  Increase the g-space decomposition by the integer factor.
  \vspace{0.15in} 
  \item \textbackslash{}phantomSym\{ on:on/off\} : \\    
  Add some extra parallelization for Smatrix computations.
  \vspace{0.15in} 
  \item \textbackslash{}useOrthoHelpers\{ off:on/off\} : \\    
  Add some extra parallelization for Lowdin orthogonalization.
  Use only if (nstates/orthograinsize)$^2<$ nproc.
  \vspace{0.15in} 
  \item \textbackslash{}invsqr\_tolerance\{1e-15 \} : \\   
  Tolerence on the iterative matrix square root method.
  \vspace{0.15in} 
  \item \textbackslash{}invsqr\_max\_iter\{10 \} : \\    
  Maximum number of iteration for the matrix square root method.
\end{enumerate}

\newpage
%-------------------------------------------------------------------
\subsection*{\bf \~{ }charm\_conf\_NL\_def}

\begin{enumerate}
  \vspace{0.15in} 
  \item \textbackslash{}numSfGrps\{ 1\} : \\    
  Decompose the $N^3$ non-local computation into the specified number of groups.
  \vspace{0.15in} 
  \item \textbackslash{}numSfDups\{ \} : \\    
  Duplicate the $N^2$ evalulation of the structure factor for the
  $N^3$ non-local computation the specified number of times.
  \vspace{0.15in} 
  \item \textbackslash{}launchNLeesFromRho\{rs:rs,rhor,rhog \} : \\  
  Specify the launch point for the Non-local EES computation.
\end{enumerate}

\newpage
%-------------------------------------------------------------------
\subsection*{\bf \~{ }charm\_conf\_map\_def}

\begin{enumerate}
  \vspace{0.15in} 
  \item \textbackslash{}Gstates\_per\_pe\{nstates \} : \\    
  If torusMap is off, specify the g-space state decomposition.
  \vspace{0.15in} 
  \item \textbackslash{}Rstates\_per\_pe\{nstates \} : \\    
  If torusMap is off, specify the r-space state decomposition.
  \vspace{0.15in} 
  \item \textbackslash{}torusMap\{on: on/off\} : \\    
  If your machine is a torus, this mapping scheme gives optimal performance.
  \vspace{0.15in} 
  \item \textbackslash{}loadMapFiles\{off:on/off \} : \\    
  Load pregenerated map files.
  \vspace{0.15in} 
  \item \textbackslash{}dumpMapFiles\{off:on/off \} : \\    
  Create map files.
  \vspace{0.15in} 
  \item \textbackslash{}fakeTorus\{off:on/off \} : \\    
  If your machine is a NOT torus, you can still use torus mapping.
  \vspace{0.15in} 
  \item \textbackslash{}torusDimX\{ nproc\} : \\    
  Fake Torus dimension along x in processors/node
  \vspace{0.15in} 
  \item \textbackslash{}torusDimY\{1 \} : \\    
  Fake Torus dimension along y in processors/node
  \vspace{0.15in} 
  \item \textbackslash{}torusDimZ\{1 \} : \\    
  Fake Torus dimension along z in processors/node
  \vspace{0.15in} 
  \item \textbackslash{}torusDimNX\{1 \} : \\    
  Fake Torus dimension along x in nodes
  \vspace{0.15in} 
  \item \textbackslash{}torusDimNY\{1 \} : \\    
  Fake Torus dimension along y in nodes
  \vspace{0.15in} 
  \item \textbackslash{}torusDimNZ\{ 1\} : \\    
  Fake Torus dimension along z in nodes
\end{enumerate}

%-------------------------------------------------------------------
%==================================================================


%==================================================================
%-----------------------------------------------------------------
\end{document}
%==================================================================

