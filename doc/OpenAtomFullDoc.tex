%==================================================================
%cccccccccccccccccccccccccccccccccccccccccccccccccccccccccccccccccc
%==================================================================

\documentstyle[12pt]{article}

\textwidth       6.9in
\headheight      0.0in
\headsep         0.0in
\topmargin      -0.3in
\textheight      9.5in
\oddsidemargin  -0.25in
\evensidemargin -0.25in
\footskip        0.25in
\parskip         6pt
\parindent       6pt
\baselineskip   12pt

\begin{document}
\large

%==================================================================
%==================================================================

\vspace*{3.in}
\begin{center}
\huge
OpenAtom Manual: \\
$\ $ \\
\Large
Modern Simulation Methods Applied to Chemistry, Physics and Biology
\end{center}

%==================================================================
%==================================================================
\clearpage
\huge
\underline{\bf Introduction}
\large

There are five types of files from which the OpenAtom
script-language interface reads commands, the
simulation setup file, the system setup file, molecular parameter/topology
files, the potential energy/pseudopotential parameter files and the
parallel decomposition parameter file. Coordinate and electronic state input files do 
not contain commands but simply free format data. 

In the simulation setup files, commands that drive a given simulation
are stated. For example, run 300 time steps of Car-Parrinello {\em ab initio}
molecular dynamics. In the system setup file, the 
system is decribed. For example, 300 water molecules, two peptides,
three counter ions, thirty Kohn-Sham states etc.  Topology files
contain information about the molecular connectitivity and potential
energy/pseudopotential files contain information about the
interactions. The parallel decomposition
input file contains all the information about the parallel decomposition.

The idea behind the file division is that one often modifies the
simulation commands, occasionally the system setup commands but seldom the 
topology, potential energy,pseudopotential or parallel decomposition
information. 
The five file types help organization and transferability. For
example, many simulation command files can drive the same system setup file, 
molecular parameter/topology files, potential energy/pseudopotential
parameter and parallel decomposition files.

There are three levels of grouping for the OpenAtom commands, 
meta-keywords, keywords and key-arguments.
The first level, the meta-keywords, sort keywords into naturally
connected groups. The second level, the keywords, are specific
commands to the computer program. The third level, key-arguments,
are the arguments to the keywords.  The syntax looks like: \\ 
\hspace*{0.5in} \~{}meta\_keyword1[ $\backslash$keyword1\{keyarg1\}
$\backslash$keyword2\{keyarg2\} ]. \\
\hspace*{0.5in} \~{}meta\_keyword2[ $\backslash$keyword1\{keyarg1\}
$\backslash$keyword2\{keyarg2\} ]. \\
The commands are case insensitive and can be specified in any order.

The present release of the code concentrates on the fine grained
parallel Car-Parrinello molecular dynamics (CPAIMD) computations 
and scaling results are given in IBM J. Res. Dev. {\bf 52} (2007). 
However, all the 
bio-widget input parameters are supported and biomolecules can be
built. Some of the biomolecular functionality is not yet implemented
in the source. The QM/MM extension is a work in progress.

In order to run the code type: \\
\hspace*{1.5in} {\bf OpenAtomMachine.x paraInfo.in simInput.in} \\
where machine is the machine-type, simInput.in is the simulation input
file and paraInfo.in is the parallel decomposition input file. 
All file names are arbitrary.

{\bf Warning:} The code will warn the user and stop the run if it isn't happy
with the input. The designers felt that {\bf not} running a
simulation was better than taking potentially inappropriate input 
and forging ahead. 
Also, the code doesn't like to overwrite files because the designers
have overwritten one too many useful files themselves and thought,  \\
\hspace*{1.25in}  A file that big?! \\
\hspace*{1.25in}  It might be very useful,\\
\hspace*{1.25in}  But now it is gone.\\
Finally, many of the error messages are sometimes written in colloquial American English, 
a product of too many late nights writing code. A language option has
not yet been implemented. The advisor may have to explain what
``Dude'' means.  However, the code does have a lot of error
checking because there is nothing worse than \\
\hspace*{1.25in}    Wind catches lily, \\
\hspace*{1.25in}    Scatt'ring petals to the wind:\\
\hspace*{1.25in}    Segmentation fault.\\
which leads one to believe \\
\hspace*{1.25in}  Errors have occurred. \\
\hspace*{1.25in}  I can't tell where or why.\\
\hspace*{1.25in}  Lazy programmers.\\
This manual was, of course, also a late night accomplishment.


%==================================================================
%==================================================================
\clearpage
\begin{center}
\huge
{\bf Simulation Keyword Dictionary: } 
\end{center}
\LARGE

The following commands maybe specified in the simulation setup file:
\begin{enumerate}
\LARGE
\item {\bf \~{}sim\_gen\_def} [ 16 keywords]
\item {\bf \~{}sim\_run\_def} [ 16 keywords]
\item {\bf \~{}sim\_nhc\_def} [ 28 keywords]
\item {\bf \~{}sim\_write\_def}[ 27 keywords]
\item {\bf \~{}sim\_list\_def} [ 14 keywords]
\item {\bf \~{}sim\_class\_PE\_def} [ 25 keywords]
\item {\bf \~{}sim\_vol\_def} [ 4 keywords]
\item {\bf \~{}sim\_cp\_def} [ 29 keywords]
\item {\bf \~{}sim\_pimd\_def} [ 10 keywords]
\end{enumerate}
\large
A complete list of all commands employed including default
values are placed in the file:
\~{}sim\_write\_def[ $\backslash$sim\_name\_def\{sim\_input.out\} ]. 


\clearpage
%==================================================================

\begin{itemize}
%-------------------------------------------------------------------
\huge
\item[] \underline{\bf \~{}sim\_gen\_def[ 16 keywords]}
\begin{enumerate}
 \vspace{0.15in} \Large
 \item $\backslash$simulation\_typ\{{\bf md},minimize,debug ...\} : \\
     \large
     Perform a molecular dynamics, minimization, pimd, cp
     cp\_pimd, cp\_wave, cp\_wave\_pimd, cp\_min, cp\_wave\_min,
     cp\_wave\_min\_pimd,
     debug, debug\_cp\_pimd, debug\_cp, debug\_pimd. Here CP indicates
     Car-Parrinello and PIMD  denotes path integral molecular dynamics.

 \vspace{0.15in} \Large
 \item  $\backslash$ensemble\_typ\{{\bf nve},nvt,npt\_i,npt\_f\} : \\
     \large
     Use the statistical ensemble specified in the curly brackets.

 \vspace{0.15in} \Large
 \item  $\backslash$minimize\_typ\{{\bf min\_std},min\_cg,min\_diis\} : \\
     \large
     Perform a minimization run using the method specified in curly brackets:
     \begin{itemize}
        \item[$\cdot$] {\it min\_std} -- Steepest descent.
        \item[$\cdot$] {\it min\_cg} -- Conjugate gradient.
        \item[$\cdot$] {\it min\_diis} -- Direct inversion in the 
                                          iterative subspace.
     \end{itemize}

 \vspace{0.15in} \Large
 \item  $\backslash$restart\_type\{{\bf initial},
             restart\_pos,restart\_posvel,restart\_all\} : \\
     \large
     Option controlling how the coordinate input file is read in:
     \begin{itemize}
       \item[$\cdot$] {\it initial} -- Start using {\it initial} format 
                                        with atomic positions in angstroms.
       \item[$\cdot$] {\it restart\_pos} -- Restart only the atomic positions.
                    Velocities and thermostat velocites (if used) are 
                    sampled from a Maxwell distribution.
       \item[$\cdot$] {\it restart\_posvel} -- Restart both atmoic position 
                     and atomic velocities.  Thermostat velocities (if used) 
                     are sampled from a Maxwell distribution.
       \item[$\cdot$] {\it restart\_all} -- Restart atomic positions, 
                     velocities, and thermostat velocities (if used).
     \end{itemize}

 \vspace{0.15in} \Large
 \item  $\backslash$num\_time\_step\{0\} : \\
     \large
     Total number of time steps for this run.

 \vspace{0.15in} \Large
 \item   $\backslash$time\_step\{1\} : \\
     \large
     Fundamental or smallest time step.  If no multiple time step integration 
     is used, then this is {\it the} time step.


 \vspace{0.15in} \Large
 \item   $\backslash$temperature\{300\} : \\
     \large
     Atomic temperature.  This is the temperature used for velocity rescaling,
     velocity resampling, thermostatting, etc.


 \vspace{0.15in} \Large
 \item   $\backslash$pressure\{0\} : \\
     \large
     Pressure of system.  Only used for simulations in the NPT ensembles.
\end{enumerate}

%-------------------------------------------------------------------
\clearpage
\huge
\item[] \underline{\bf \~{}sim\_run\_def[ 16 keywords]}
\begin{enumerate}
 \vspace{0.15in} \Large
 \item   $\backslash$init\_resmp\_atm\_vel\{on,{\bf off}\} : \\
     \large
     If turned on, atomic velocities are initially sampled/resampled from a 
     Maxwell distribution.


 \vspace{0.15in} \Large
 \item   $\backslash$resmpl\_frq\_atm\_vel\{0\} : \\
     \large
     Atomic velocities are resampled with a frequency equal to the number 
     in the curly brackets.  A zero indicates no resampling is to be done.


 \vspace{0.15in} \Large
 \item   $\backslash$init\_rescale\_atm\_vel\{on,{\bf off}\} : \\
    \large
    If turned on, atomic velocities are initially rescaled to preset 
    temperature.

 \vspace{0.15in} \Large
 \item   $\backslash$rescale\_frq\_atm\_vel\{0\} : \\
    \large
    Atomic velocities are rescaled to desired temperature
    with a frequency equal to the number in the
    curly brackets.  A zero indicates no resampling is to be done.

 \vspace{0.15in} \Large
 \item   $\backslash$init\_rescale\_atm\_nhc\{on,{\bf off}\} : \\
    \large
    If turned on, atomic thermostat velocities are initially rescaled to 
    preset temperature.

 \vspace{0.15in} \Large
 \item   $\backslash$zero\_com\_vel\{yes,{\bf no}\} : \\
    \large
    If set to yes, the center of mass velocity is initially set to 0.

 \vspace{0.15in} \Large
 \item   $\backslash$respa\_steps\_lrf\{0\} : \\
    \large
    Number of short time steps to be taken between long range force updates.
    A zero indicates long range forces are updated every step.

 \vspace{0.15in} \Large
 \item   $\backslash$respa\_rheal\{1\} : \\
   \large
    If $\backslash$respa\_steps\_lrf is not zero, then the long range 
    and short range forces are both switched off in space with a switching
    function of healing length given in the curly brackets.

 \vspace{0.15in} \Large
 \item   $\backslash$respa\_steps\_torsion\{0\} : \\
   \large
    Number of short time steps to be taken between updates of torsional
    forces.  A zero indicates torsional forces are updated every step.

 \vspace{0.15in} \Large
  \item   $\backslash$respa\_steps\_intra\{0\} : \\
   \large
    Number of short time steps to be taken between updates of 
    {\it intermolcular} forces.  A zero indicates intermolecular forces are 
    updated every step.

 \vspace{0.15in} \Large
 \item   $\backslash$shake\_tol\{1.0e-6\} : \\
    \large
     Constraints not treated by the group constraint method are iterated to a
     tolerance given in curly brackets.


 \vspace{0.15in} \Large
 \item   $\backslash$rattle\_tol\{1.0e-6\} : \\
    \large
      Time derivative of constraints not treated by the group constraint 
      method are iterated to a tolerance given in curly brackets.


 \vspace{0.15in} \Large
 \item   $\backslash$max\_constrnt\_iter\{200\} : \\
    \large
     Maximum number of iterations to be performed on constraints before a 
     warning that the tolerance has not yet been reached is printed out and 
     iteration stops.

 \vspace{0.15in} \Large
 \item   $\backslash$group\_con\_tol\{1.0e-6\} : \\
    \large
     Group constraint method is iterated to a tolerance given in curly 
     brackets.

\end{enumerate}

%-------------------------------------------------------------------
\clearpage
\huge
\item[] \underline{\bf \~{}sim\_nhc\_def[ 28 keywords]}
\begin{enumerate}
 \vspace{0.15in} \Large
 \item   $\backslash$atm\_nhc\_tau\_def\{1000\} : \\
 \large
  Time scale (in femtoseconds) on which the thermostats should evolve. 
  Generally set to some characteristic time scale in the system.

 \vspace{0.15in} \Large
 \item   $\backslash$init\_resmp\_atm\_nhc\{on,{\bf off}\} : \\
  \large
   If turned on, atomic thermostat velocities will be initially resampled from
   a Maxwell distribution.

 \vspace{0.15in} \Large
 \item   $\backslash$atm\_nhc\_len\{2\} : \\
  \large
   Number of elements in the thermostat chain.  Generally 2-4 is adequate.

 \vspace{0.15in}\Large
 \item   $\backslash$respa\_steps\_nhc\{2\} : \\
   \large
     Number of individual Suzuki/Yoshida factorizations to be employed in
     integration of the thermostat variables.  Only
     increase if energy conservation not satisfactory with default value.

 \vspace{0.15in}\Large
 \item   $\backslash$yosh\_steps\_nhc\{1,{\bf 3},5,7\} : \\
  \large
   Order of the Suzuki/Yoshida integration scheme for the thermostat
   variables.  Only increase if energy conservation not satisfactory with
   default value.

 \vspace{0.15in} \Large
 \item   $\backslash$resmpl\_atm\_nhc\{0\} : \\
  \large
   Atomic thermostat velocities will be resampled with a frequency specified
   in the curly brackets.  A zero indicates no resampling of thermostat
   velocities.

 \vspace{0.15in}\Large
 \item   $\backslash$respa\_xi\_opt\{{\bf 1},2,3,4\} : \\
 \large
   Depth of penetration of thermostat integration into multiple time step
   levels.  Larger numbers indicate deeper penetration.  A one indicates
   thermostat variables are updated at the beginning and end of 
   every step (XO option).

 \vspace{0.15in}\Large
 \item   $\backslash$atm\_isokin\_opt\{on:on/off\}: \\ \large
    Thermostat the atoms using isokinetic NHC (on) or plain NHC (off).

 \vspace{0.15in}\Large
 \item   $\backslash$cp\_thermstats\{on:on/off\}: \\ \large
    Thermostat the electrons (on) or run them without control (off).
    If ``on'', at least one isokinetic octopus NHC thermostat is placed on each
    g-space plane of each state.

 \vspace{0.15in}\Large
 \item   $\backslash$cp\_num\_nhc\_iso\{2\}: \\ \large
    The number of tenticles in an isokinetic octopus NHC thermostat.

 \vspace{0.15in}\Large
 \item   $\backslash$cp\_nhc\_chunk\{1\}: \\ \large
    The number of thermostats per g-space electronic state plane.
\end{enumerate}

%-------------------------------------------------------------------
\clearpage
\huge
\item[] \underline{\bf \~{}sim\_write\_def[ 27 keywords]}
\begin{enumerate}

 \vspace{0.15in} \Large
 \item   $\backslash$sim\_name\{sim\_input.out\} : \\
    A complete list of all commands specified and defaults used.

 \vspace{0.15in} \Large
 \item   $\backslash$write\_screen\_freq\{1\} : \\
  \large
   Output information to screen will be written with a frequency specified
   in curly brackets.

 \vspace{0.15in} \Large
 \item   $\backslash$write\_dump\_freq\{1\} : \\
   \large
     The restart file of atomic coordinates and velocities, etc. will be
     written with a frequency specified in curly brackets.

 \vspace{0.15in}\Large
 \item   $\backslash$write\_inst\_freq\{1\} : \\
   \large
     Instantaneous averages of energy, temperature, pressure, etc. will be 
     written to the instantaneous file with a frequency specified in curly 
     brackets.

 \vspace{0.15in} \Large
 \item   $\backslash$write\_pos\_freq\{100\} : \\
  \large
    Atomic positions will be appended to the trajectory position file with
    a frequency specified in curly brackets.

 \vspace{0.15in} \Large
 \item   $\backslash$write\_vel\_freq\{100\} : \\
  \large
    Atomic velocities will be appended to the trajectory velocity file with
    a frequency specified in curly brackets.

 \vspace{0.15in} \Large
 \item   $\backslash$sim\_name\{sim\_input.out\} : \\
   \large
     Name of file containing all keywords, both user set a default.

 \vspace{0.15in} \Large
 \item   $\backslash$out\_restart\_file\{sim\_restart.out\} : \\
   \large Name of the restart (dump) file containing final atomic 
          coordinates and velocities.

 \vspace{0.15in} \Large
 \item   $\backslash$in\_restart\_file\{sim\_restart.in\} : \\
   \large
     Name of the file containing initial atomic coordinates and
     velocities for this run.


 \vspace{0.15in} \Large
 \item   $\backslash$instant\_file\{sim\_instant.out\} : \\
   \large
     File to which instantaneous averages are to be written.

 \vspace{0.15in}\Large
 \item   $\backslash$atm\_pos\_file\{sim\_atm\_pos.out\} : \\
   \large
     Trajectory position file name.

 \vspace{0.15in} \Large
 \item   $\backslash$atm\_vel\_file\{sim\_atm\_vel.out\} : \\
  \large Trajectory force file name.  

 \vspace{0.15in} \Large
 \item   $\backslash$atm\_force\_file\{sim\_atm\_force.out\} : \\
   \large
     Trajectory velocity file name.

 \vspace{0.15in}
 \Large
 \item   $\backslash$conf\_partial\_file\{sim\_atm\_pos\_part.out\} : \\
   \large
     Partial trajectory position file name.

 \vspace{0.15in} \Large
 \item   $\backslash$mol\_set\_file\{sim\_atm\_mol\_set.in\} : \\
   \large
    Name of file containing instructions on how to name and build molecules.
    Detailed instructions on contents of this file given elsewhere.

 \vspace{0.15in} \Large
 \item   $\backslash$write\_force\_freq\{1000000\} : \\
   \large
   Atomic forces are written to the trajectory force file with a
   frequency specified in the curly brackets.

 \vspace{0.15in} \Large
 \item   $\backslash$screen\_output\_units\{{\bf au},kcal\_mol,kelvin\} : \\
   \large
   Output information is written to the screen in units specified 
   in curly brackets.

 \vspace{0.15in} \Large
 \item   $\backslash$conf\_file\_format\{{\bf binary},formatted\} : \\
   \large
   Trajectory files are written either in binary or formatted form.

 \vspace{0.15in} \Large
 \item   $\backslash$conf\_partial\_limits\{1,0\} : \\
   \large
   Two numbers specifying the first and last atoms to be written to
   the partial trajectory position file.

\end{enumerate}

%-------------------------------------------------------------------
\clearpage
\huge
\item[] \underline{\bf \~{}sim\_list\_def[ 14 keywords]}
\begin{enumerate}

 \vspace{0.15in} \Large
 \item   $\backslash$neighbor\_list\{no\_list,{\bf ver\_list},lnk\_list\} : \\
   \large
   Type of neighbor list to use.  Optimal scheme for most systems: 
   $\backslash$ver\_list with a $\backslash$lnk\_lst update type.

 \vspace{0.15in} \Large
 \item   $\backslash$update\_type\{{\bf lnk\_lst},no\_list\} : \\
   \large
   Update the verlet list using either no list or a link list.

 \vspace{0.15in} \Large
 \item   $\backslash$verlist\_skin\{1\} : \\
   \large
   Skin depth added to verlet list cutoff.  Needs to be optimized between
   extra neighbors added and average required update frequency.

 \vspace{0.15in} \Large
 \item   $\backslash$lnk\_cell\_divs\{7\} : \\
  \large
   Number of link cell divisions in each dimension.  Needs to be optimized
   if used either as list type or update type.

\end{enumerate}

%-------------------------------------------------------------------
\clearpage
\huge
\item[] \underline{\bf \~{}sim\_class\_PE\_def[ 25 keywords]}
\begin{enumerate}

 \vspace{0.15in} \Large
 \item   $\backslash$ewald\_kmax\{7\} : \\
  \large
  Maximum length of k-vectors used in evaluating Ewald summation for 
  electrostatic interactions.

 \vspace{0.15in} \Large
 \item   $\backslash$ewald\_alpha\{7\} : \\
  \large
  Size of real-space damping (screening) parameter in Ewald sum.

 \vspace{0.15in} \Large
 \item   $\backslash$ewald\_respa\_kmax\{0\} : \\
  \large
   Maximum length of k-vectors used in long range force RESPA integration.

 \vspace{0.15in}\Large
 \item   $\backslash$ewald\_pme\_opt\{on,{\bf off}\} : \\
  \large
  If turned on, then electrostatic interactions are evaluated using the
  smooth particle mesh method.  Its use is recommended.

 \vspace{0.15in} \Large
 \item   $\backslash$ewald\_kmax\_pme\{7\} : \\
  \large
  Maximum length of k-vectors used in evaluating Ewald summation for 
  electrostatic interactions via the smooth particle mesh method.

 \vspace{0.15in} \Large
 \item   $\backslash$ewald\_interp\_pme\{4\} : \\
   \large
   Order of mesh interpolation in the smooth particle mesh Ewald method.

 \vspace{0.15in} \Large
 \item   $\backslash$ewald\_respa\_pme\_opt\{on,{\bf off}\} : \\
  \large
  If turned on, then long range force RESPA is used in conjunction with the
  smooth particle mesh method.

 \vspace{0.15in} \Large
 \item   $\backslash$ewald\_respa\_interp\_pme\{4\} : \\
  \large
  Order of mesh interpolation in the smooth particle mesh Ewald method used
  with RESPA.

 \vspace{0.15in}\Large
 \item   $\backslash$sep\_VanderWaals\{on,{\bf off}\} : \\
   \large
   If turned on, then the VanderWaals component of the {\it inter}molecular 
   interaction energy is printed separately to screen.

 \vspace{0.15in} \Large
 \item   $\backslash$inter\_spline\_pts\{2000\} : \\
   \large
    Number of spline points used {\it inter}molecular interaction 
    potential and derivative.

\end{enumerate}


%-------------------------------------------------------------------
\clearpage
\huge
\item[] \underline{\bf \~{}sim\_vol\_def[ 4 keywords]}
\begin{enumerate}

 \vspace{0.15in} \Large
 \item  $\backslash$volume\_tau\{1000\} : \\
    \large
     Time scale of volume evolution
     under constant pressure (NPT\_I or NPT\_F) in femtoseconds.
  
 \vspace{0.15in} \Large
 \item  $\backslash$volume\_nhc\_tau\{1000\} : \\
    \large
     Time scale of volume 
     heat bath evolution under constant pressure in femtoseconds.

\end{enumerate}

%-------------------------------------------------------------------
\clearpage
\huge
\item[] \underline{\bf \~{}sim\_cp\_def[ 29 keywords]}
\begin{enumerate}

 \vspace{0.15in} \Large
 \item  $\backslash$cp\_e\_e\_interact\{{\bf on},off\} : \\
    \large
     Electron-electron interactions can be turned off for 1e problems.

 \vspace{0.15in} \Large
 \item  $\backslash$cp\_dft\_typ\{{\bf lda},lsda,gga\_lda,gga\_lsda\} : \\
    \large
     Density function type where gga mean gradient corrections are on.

 \vspace{0.15in} \Large
 \item  $\backslash$cp\_vxc\_typ\{{\bf pz\_lda},pw\_lda,pz\_lsda\} : \\
    \large
     Exchange correlation function type.

 \vspace{0.15in} \Large
 \item  $\backslash$cp\_ggax\_typ\{becke,pw91x,fila\_1x,fil\_2x,{\bf off}\} : \\
    \large
     Gradient corrected exchange function type.

 \vspace{0.15in} \Large
 \item  $\backslash$cp\_ggac\_typ\{lyp,lypm1,pw91c,{\bf off}\} : \\
    \large
     Gradient corrected correlation function type.

 \vspace{0.15in} \Large
 \item  $\backslash$cp\_norb\{full\_ortho,norm\_only,{\bf off}\} : \\
    \large
     Non-orthogonal orbitals option (see Tuckerman-Hutter-Parr. papers).

 \vspace{0.15in} \Large
 \item  $\backslash$cp\_minimize\_typ\{{\bf min\_std},min\_cg\} : \\
    \large
      Minimization type, steepest descent, conjugate gradient 

 \vspace{0.15in} \Large
 \item  $\backslash$cp\_mass\_tau\_def\{25\} : \\
    \large
      CP fictitious dynamics time scale in femtoseconds.

 \vspace{0.15in} \Large
 \item  $\backslash$cp\_energy\_cut\_def\{2\} : \\
    \large
      Plane wave expansion cutoff $E_g<=\hbar^2g^2/2m_e$ 

 \vspace{0.15in} \Large
 \item  $\backslash$cp\_mass\_cut\_def\{2\} : \\
    \large
      Renormalize CP masses for $E_g<=\hbar^2g^2/2m_e$ Rhydberg.

 \vspace{0.15in} \Large
 \item  $\backslash$cp\_fict\_KE\{1000\} : \\
    \large
      Energy give to the plane waves to create the Born-Oppenheimer
      surface with there fake dynamics

 \vspace{0.15in} \Large
 \item  $\backslash$cp\_ptens\{on,{\bf off}\} : \\
    \large
      Evaluate the pressure tensor. 

 \vspace{0.15in} \Large
 \item  $\backslash$cp\_init\_orthog\{on,{\bf off}\} : \\
    \large
      Orthogonalize the states at the start.

 \vspace{0.15in} \Large
 \item  $\backslash$cp\_orth\_meth\{{\bf gram\_schmidt},lowdin\} : \\
    \large
      Orthogonalize using gram-schmidt or lowdin.

 \vspace{0.15in} \Large
 \item  $\backslash$cp\_restart\_type\{{\bf gen\_wave},initial,
                         restart\_pos,restart\_posvel,restart\_all\} : \\
    \large
      Start the run by generating sum guess to the states, ...

 \vspace{0.15in} \Large
 \item  $\backslash$cp\_nonloc\_ees\_opt\{on:on/off \}: \\     \large
 Compute the non-local pseudopotential energy and forces using Euler
 exponential spline interpolation (EES).

 \vspace{0.15in} \Large
 \item  $\backslash$cp\_eext\_ees\_opt\{on:on/off \}: \\     \large
     Compute the local pseudopotential energy and forces as well as the 
     reciprocal space part of the Ewald sum using  EES.

 \vspace{0.15in} \Large
 \item  $\backslash$cp\_pseudo\_ees\_order\{6 \}: \\     \large
     Use a EES interpolation order of 6.

 \vspace{0.15in} \Large
 \item  $\backslash$cp\_pseudo\_ees\_scale\{1.4 \}: \\     \large
     Use a g-space expanded 1.4x to perform the EES interpolation.
\end{enumerate}

%-------------------------------------------------------------------
\clearpage
\huge
\item[] \underline{\bf \~{}sim\_pimd\_def[ 10 keywords]}
\begin{enumerate}

 \vspace{0.15in} \Large
 \item  $\backslash$path\_int\_beads\{1\} : \\
    \large
     Number of path integral beads.

 \vspace{0.15in} \Large
 \item  $\backslash$path\_int\_md\_typ\{staging,centroid\} : \\
    \large
     Path integral molecular dynamics type, staging or centroid/normal mode.
     Staging gives the fastest equilibration.

 \vspace{0.15in} \Large
 \item  $\backslash$path\_int\_gamma\_adb\{1\} : \\
    \large
     Path integral molecular dynamics type adiabaticity parameter.
     Employed with the centroid option to give approx. to quantum dynamics.

 \vspace{0.15in} \Large
 \item  $\backslash$respa\_steps\_pimd\{1\} : \\
    \large
     Path integral bead respa steps.

 \vspace{0.15in} \Large
 \item  $\backslash$initial\_spread\_size\{1\} : \\
    \large
    In growing cyclic initial paths, what is the how large do you
    want their diameter or spread in Angstroms.

 \vspace{0.15in} \Large
 \item  $\backslash$initial\_spread\_opt\{on,{\bf off}\} : \\
    \large
    Do you want initial paths grown for you?

\end{enumerate}

%----------------------------------------------------------------
\end{itemize}
%==================================================================


%==================================================================
%==================================================================
\clearpage
\begin{center}
\huge
{\bf Molecule and Wave function \\
Keyword Dictionary: } 
\end{center}

The following commands maybe specified in the system setup file:
\begin{enumerate}
\LARGE
\item {\bf \~{}molecule\_def []}
\item {\bf \~{}wavefunc\_def []}
\item {\bf \~{}bond\_free\_def []}
\item {\bf \~{}data\_base\_def []}
\end{enumerate}

%==================================================================
\begin{itemize}

%-------------------------------------------------------------------
\clearpage
\huge
\item[] \underline{\bf \~{}molecule\_def}
\begin{enumerate}

 \vspace{0.15in} \Large
 \item  $\backslash$mol\_name\{\} : \\
    \large
   The name of the molecule type.

 \vspace{0.15in} \Large
 \item  $\backslash$num\_mol\{ \} : \\ 
    \large
   The number of this molecule type you want in your system.

 \vspace{0.15in} \Large
 \item  $\backslash$num\_residue\{ \} : \\
    \large
   The number of residues in in this molecule type.

 \vspace{0.15in} \Large
 \item  $\backslash$mol\_index\{2\} : \\ 
    \large
   This is the ``2nd'' molecule type defined in the system.

 \vspace{0.15in} \Large
 \item  $\backslash$mol\_parm\_file\{ \} : \\
    \large
   The topology of the molecule type is described in this file.

 \vspace{0.15in} \Large
 \item  $\backslash$mol\_text\_nhc\{ \} : \\
    \large
   The temperature of this molecule type. It may be different than the
   external temperature for fancy simulation studies in the adiabatic limit.

 \vspace{0.15in} \Large
 \item  $\backslash$mol\_freeze\_opt\{{\bf none},all,backbone\} : \\
    \large
    Freeze this molecule to equilibrate the system.

 \vspace{0.15in} \Large
 \item  $\backslash$hydrog\_mass\_opt\{A,B\} : \\
    \large
   Increase the mass of type A hydrogens where A is 
   {\bf off},all,backbone or sidechain to B amu. This allows
   selective deuteration for example.

 \vspace{0.15in} \Large
 \item  $\backslash$hydrog\_con\_opt\{{\bf off},all,polar\} : \\
    \large
   Constrain all bonds to this type of hydrogen in the molecule.

 \vspace{0.15in} \Large
 \item  $\backslash$mol\_nhc\_opt\{none,{\bf global},glob\_mol,ind\_mol,\\ 
                 \hspace*{0.5in}res\_mol,atm\_mol,mass\_mol\} :  \\
    \large
   Nose-Hoover Chain option: Couple the atoms in this molecule to:
      \begin{itemize}
         \item none:     no thermostats.
         \item global:   the global thermostat.
         \item glob\_mol: a thermostat for this molecule type. 
         \item ind\_mol:  a thermostat for each molecule of this type. 
         \item res\_mol:  a thermostat for each residue in each molecule.
         \item atm\_mol:  a thermostat for each atom in each molecule.
         \item mass\_mol: a thermostat for each degree of freedom.
      \end{itemize}

 \vspace{0.15in} \Large
 \item  $\backslash$mol\_tau\_nhc\{ \} : \\
    \large
   Nose-Hoover Chain time scale in femtoseconds for this molecule type.

\end{enumerate}

%-------------------------------------------------------------------
\clearpage
\huge
\item[] \underline{\bf \~{}wavefunc\_def}
\begin{enumerate}


 \vspace{0.15in} \Large
 \item  $\backslash$nstate\_up\{1\} : \\ 
    \large
     Number of spin up states in the spin up electron density.

 \vspace{0.15in} \Large
 \item  $\backslash$nstate\_dn\{1\} : \\
    \large
     Number of spin down states in the spin down electron density.

 \vspace{0.15in} \Large
 \item  $\backslash$cp\_tau\_nhc\{25\} : \\
    \large
     Nose-Hoover Chain coupling time scale.

\end{enumerate}

%-------------------------------------------------------------------
\clearpage
\huge
\item[] \underline{\bf \~{}bond\_free\_def}
\begin{enumerate}

 \vspace{0.15in} \Large
 \item  $\backslash$atom1\_moltyp\_ind\{ \} : \\ 
    \large
    Index of molecule type to which atom 1 belongs.

 \vspace{0.15in} \Large
 \item  $\backslash$atom2\_moltyp\_ind\{ \} : \\ 
    \large
    Index of molecule type to which atom 2 belongs.

 \vspace{0.15in} \Large
 \item  $\backslash$atom1\_mol\_ind\{ \} : \\    
    \large
    Index of the molecule of the specified molecule type 
    to which atom 1 belongs.

 \vspace{0.15in} \Large
 \item  $\backslash$atom2\_mol\_ind\{ \} : \\    
    \large
    Index of the molecule of the specified molecule type 
    to which atom 2 belongs.

 \vspace{0.15in} \Large
 \item  $\backslash$atom1\_residue\_ind\{ \} : \\ 
    \large
    Index of the residue in the molecule to which atom 1 belongs.

 \vspace{0.15in} \Large
 \item  $\backslash$atom2\_residue\_ind\{ \} : \\ 
    \large
    Index of the residue in the molecule to which atom 2 belongs.

 \vspace{0.15in} \Large
 \item  $\backslash$atom1\_atm\_ind\{ \} : \\    
    \large
    Index of the atom 1 in the residue.
 
 \vspace{0.15in} \Large
 \item  $\backslash$atom2\_atm\_ind\{ \} : \\     
    \large
    Index of the atom 2 in the residue.
 
 \vspace{0.15in} \Large
 \item  $\backslash$eq\{ \} : \\               
    \large
    Equilibrium bond length in Angstrom.
  
 \vspace{0.15in} \Large
 \item  $\backslash$fk\{ \} : \\               
    \large
    Umbrella sampling force constant in K/Angstro${\rm m}^2$.

 \vspace{0.15in} \Large
 \item  $\backslash$rmin\_hist\{ \} : \\         
    \large
    Min Histogram distance in Angstrom.

 \vspace{0.15in} \Large
 \item  $\backslash$rmax\_hist\{ \} : \\         
    \large
    Max Histogram distance in Angstrom.

 \vspace{0.15in} \Large
 \item  $\backslash$num\_hist\{ \} : \\         
    \large
    Number of points in the histogram.
 
 \vspace{0.15in} \Large
 \item  $\backslash$hist\_file\{ \} : \\        
    \large
    File to which the histogram is printed.

\end{enumerate}

%-------------------------------------------------------------------
\clearpage
\huge
\item[] \underline{\bf \~{}data\_base\_def}
\begin{enumerate}

 \vspace{0.15in} \Large
 \item  $\backslash$inter\_file\{ \} : \\
    \large
    File containing intermolecular interaction parameters.

 \vspace{0.15in} \Large
 \item  $\backslash$vps\_file\{pi\_md.inter\} : \\
    \large
    File containing pseudopotential interaction parameters.

 \vspace{0.15in} \Large
 \item  $\backslash$bond\_file\{pi\_md.bond\} : \\
    \large
    File containing bond interaction parameters.

 \vspace{0.15in} \Large
 \item  $\backslash$bend\_file\{pi\_md.bend\} : \\
    \large
    File containing bend interaction parameters.

 \vspace{0.15in} \Large
 \item  $\backslash$tors\_file\{pi\_md.tors\} : \\
    \large
    File containing torsion interaction parameters.

 \vspace{0.15in} \Large
 \item  $\backslash$onefour\_file\{pi\_md.onfo\} : \\
    \large
    File containing onefour interaction parameters.

\end{enumerate}

%-------------------------------------------------------------------
\end{itemize}
%==================================================================


%==================================================================
%==================================================================
\clearpage
\begin{center}
\huge
{\bf Topology Keyword Dictionary: } 
\end{center}
\large

The following commands maybe specified in the molecular topology 
and parameter files:
\begin{enumerate}
\LARGE
\item {\bf \~{ }molecule\_name\_def []}
\item {\bf \~{ }residue\_def []}
\item {\bf \~{ }residue\_bond\_def []}
\item {\bf \~{ }residue\_name\_def []}
\item {\bf \~{ }atom\_def []} 
\item {\bf \~{ }bond\_def []}
\item {\bf \~{ }grp\_bond\_def []}
\item {\bf \~{ }bend\_def []}
\item {\bf \~{ }bend\_bnd\_def []}
\item {\bf \~{ }tors\_def []}
\item {\bf \~{ }onfo\_def   []}   
\item {\bf \~{ }residue\_morph []}
\item {\bf \~{ }atom\_destroy []}
\item {\bf \~{ }atom\_create []}
\item {\bf \~{ }atom\_morph []}
\end{enumerate}
%==================================================================

%==================================================================
\begin{itemize}

%-------------------------------------------------------------------
\clearpage
\huge
\item[] {\bf \~{ }molecule\_name\_def []}
\begin{enumerate}

 \vspace{0.15in} \Large
 \item  $\backslash$molecule\_name\{\} : \\ 
   \large
    Molecule name in characters.

 \vspace{0.15in} \Large
 \item  $\backslash$nresidue\{\} : \\ 
   \large
    Number of residues in the molecule.

 \vspace{0.15in} \Large
 \item  $\backslash$natom\{\} : \\ 
   \large
    Number of atoms in the molecule.
\end{enumerate}

%-------------------------------------------------------------------
\clearpage
\huge
\item [] {\bf \~{ }residue\_def []}
\begin{enumerate}

 \vspace{0.15in} \Large
 \item  $\backslash$residue\_name\{\} : \\ 
    Residue name in characters.

 \vspace{0.15in} \Large
 \item  $\backslash$residue\_index\{\} : \\ 
   \large
    Residue index number (this is the third residue in the molecule).

 \vspace{0.15in} \Large
 \item  $\backslash$natom\{\} : \\ 
   \large
    Number of atoms in the residue (after morphing).

 \vspace{0.15in} \Large
 \item  $\backslash$residue\_parm\_file\{\} : \\ 
   \large
    Residue parameter file.

 \vspace{0.15in} \Large
 \item  $\backslash$residue\_fix\_file\{\} : \\ 
   \large
    File containing morphing instructions for this residue if any.

\end{enumerate}

%-------------------------------------------------------------------
\clearpage
\huge
\item [] {\bf \~{ }residue\_bond\_def []}
\begin{enumerate}

 \vspace{0.15in} \Large
 \item  $\backslash$res1\_typ\{\} : \\ 
   \large
    Residue name in characters.

 \vspace{0.15in} \Large
 \item  $\backslash$res2\_typ\{\} : \\ 
   \large
    Residue name in characters.

 \vspace{0.15in} \Large
 \item  $\backslash$res1\_index\{\} : \\ 
   \large
    Numerical index of first residue.

 \vspace{0.15in} \Large
 \item  $\backslash$res2\_index\{\} : \\ 
   \large
    Numerical index of second residue.

 \vspace{0.15in} \Large
 \item  $\backslash$res1\_bond\_site\{\} : \\ 
   \large
   Connection point of residue bond in residue 1.

 \vspace{0.15in} \Large
 \item  $\backslash$res2\_bond\_site\{\} : \\ 
   \large
   Connection point of residue bond in residue 2.

 \vspace{0.15in} \Large
 \item  $\backslash$res1\_bondfile\{\} : \\ 
   \large
   Morph file for residue 1 (lose a hydrogen, for example).

 \vspace{0.15in} \Large
 \item  $\backslash$res2\_bondfile\{\} : \\ 
   \large
   Morph file for residue 2 (lose an halogen atom, for example).
  
\end{enumerate}

%-------------------------------------------------------------------
\clearpage
\huge
\item [] {\bf \~{ }residue\_name\_def []}
\begin{enumerate}

 \vspace{0.15in} \Large
 \item  $\backslash$res\_name\{\} : \\ 
   \large
   Name of the residue in characters.

 \vspace{0.15in} \Large
 \item  $\backslash$natom\{\} : \\ 
   \large
   Number of atoms in the residue
\end{enumerate}

%-------------------------------------------------------------------
\clearpage
\huge
\item [] {\bf \~{ }atom\_def []} 
\begin{enumerate}

 \vspace{0.15in} \Large
 \item  $\backslash$atom\_typ\{\} : \\ 
   \large
   Type of atom in characters.
 
 \vspace{0.15in} \Large
 \item  $\backslash$atom\_ind\{\} : \\ 
   \large
   This is the nth atom in the molecule. The number is used to 
   define bonds, bends, tors, etc involving this atom.

 \vspace{0.15in} \Large
 \item  $\backslash$mass\{\} : \\ 
   \large
    The mass of the atom in amu.

 \vspace{0.15in} \Large
 \item  $\backslash$charge\{\} : \\ 
   \large
    The charge on the atom in ``e''.

 \vspace{0.15in} \Large
 \item  $\backslash$valence\{\} : \\ 
   \large
    The valence of the atom.

 \vspace{0.15in} \Large
 \item  $\backslash$improper\_def\{0,1,2,3\} : \\ 
   \large
    How does this atom like its improper torsion constructed (if any).

 \vspace{0.15in} \Large
 \item  $\backslash$bond\_site\_1\{site,prim-branch,sec-branch\} : \\ 
   \large
    To what bond-site(s), character string, does this atom belong and where is
    it in the topology tree structure relative to the root (the atom
    that will bonded to some incoming atom).
      \begin{itemize}
        \item Atom to be bonded =root atom  (0,0).
        \item 1st atom bonded to root atom  (1,0).
        \item 2nd atom bonded to root atom  (2,0).
        \item 1st atom bonded to 1st branch (1,1).
        \item 2nd atom bonded to 1st branch (1,2).
      \end{itemize}


 \vspace{0.15in} \Large
 \item  $\backslash$def\_ghost1\{index,coeff\} : \\ 
   \large
    If the atom is a ghost (these funky sites that TIP4P type
    models have), it spatial position is constructed by taking linear 
    combinations of the other atoms in the molecule specified by the index 
    and the coef. A ghost can be composed of up to six other atoms
    (ghost\_1{} ... ghost\_6{}).

\end{enumerate}

%-------------------------------------------------------------------
\clearpage
\huge
\item [] {\bf \~{ }bond\_def []}
\begin{enumerate}
 \vspace{0.15in} \Large
 \item  $\backslash$atom1\{\} : \\
   \large
   Numerical index of atom 1.

 \vspace{0.15in} \Large
 \item  $\backslash$atom2\{\} : \\
   \large
   Numerical index of atom 2.

 \vspace{0.15in} \Large
 \item  $\backslash$modifier\{con,{\bf on},off\} : \\
   \large
   The bond is active=on, inactive=off, or constrained=con.

\end{enumerate}
%-------------------------------------------------------------------
\clearpage
\huge
\item [] {\bf \~{ }residue\_morph []}
\begin{itemize}
 \large \item Used to define a file which contains modifications to 
              a given residue.
 \large \item It has the same arguments as \~{ }residue\_name\_def[].
\end{itemize}
%-------------------------------------------------------------------
\huge
\item [] {\bf \~{ }atom\_destroy []}
\begin{itemize}
 \large  \item Used to destroy an atom in a residue morph file.
 \large  \item Uses the same key words as \~{ }atom\_def[]
\end{itemize}
%-------------------------------------------------------------------
\huge
\item [] {\bf \~{ }atom\_create []}
\begin{itemize}
 \large  \item Used to create an atom in a residue morph file.
 \large  \item Uses the same key words as \~{ }atom\_def[]
\end{itemize}
%-------------------------------------------------------------------
\huge
\item [] {\bf \~{ }atom\_morph []}
\begin{itemize}
 \large  \item Used to morph an atom in a residue morph file.
 \large  \item Uses the same key words as \~{ }atom\_def[]
\end{itemize}

%-------------------------------------------------------------------
\end{itemize}
%==================================================================


%==================================================================
%==================================================================
\clearpage
\begin{center}
\huge
{\bf Potential Parameter \\ Keyword Dictionary: } 
\end{center}
\large

The following commands maybe specified in the potential parameter files:
\begin{enumerate}
\LARGE
\item {\bf \~{ }inter\_parm []}
\item {\bf \~{ }bond\_parm []}
\item {\bf \~{ }bend\_parm []}
\item {\bf \~{ }tors\_parm []}
\item {\bf \~{ }onfo\_parm []}
\item {\bf \~{ }pseudo\_parm []}
\item {\bf \~{ }bend\_bnd\_parm []}
\end{enumerate}

%==================================================================


%==================================================================
\begin{itemize}

%-------------------------------------------------------------------
\clearpage
\huge
\item[] \underline{\bf \~{}inter\_parm}
\begin{enumerate}

 \vspace{0.15in} \Large
 \item  $\backslash$atom1\{\} : \\ 
    \large
    Atom type 1 (character data).

 \vspace{0.15in} \Large
 \item  $\backslash$atom2\{\} : \\ 
    \large
    Atom type 2 (character data).

 \vspace{0.15in} \Large
 \item  $\backslash$pot\_type\{lennard-jones,williams,aziz-chen,null\} : \\ 
    \large
    Potential type: Williams is an exponential c6-c8-c10. Aziz-chen has
    a short range switching function on the vanderwaals part.

 \vspace{0.15in} \Large
 \item  $\backslash$min\_dist\{\} : \\ 
    \large
    Minimum interaction distance.

 \vspace{0.15in} \Large
 \item  $\backslash$max\_dist\{\} : \\ 
    \large
    Spherical cutoff interaction distance.

 \vspace{0.15in} \Large
 \item  $\backslash$res\_dist\{\} : \\ 
    \large
    Respa Spherical cutoff interaction distance.

 \vspace{0.15in} \Large
 \item  $\backslash$sig\{\} : \\ 
    \large
    Lennard-Jones parameter.

 \vspace{0.15in} \Large
 \item  $\backslash$eps\{\} : \\ 
    \large
    Lennard-Jones parameter.

 \vspace{0.15in} \Large
 \item  $\backslash$c6\{\} : \\ 
    \large
    Williams/Aziz-Chen Vdw parameter.

 \vspace{0.15in} \Large
 \item  $\backslash$c8\{\} : \\ 
    \large
    Williams/Aziz-Chen Vdw parameter.

 \vspace{0.15in} \Large
 \item  $\backslash$c9\{\} : \\ 
    \large
    Williams/Aziz-Chen Vdw parameter.

 \vspace{0.15in} \Large
 \item  $\backslash$c10\{\} : \\ 
    \large
    Williams/Aziz-Chen Vdw parameter.

 \vspace{0.15in} \Large
 \item  $\backslash$Awill\{\} : \\ 
    \large
    Williams/Aziz-Chen parameter ($A\exp(-Br-Cr^2)$).

 \vspace{0.15in} \Large
 \item  $\backslash$Bwill\{\} : \\ 
    \large
    Williams/Aziz-Chen parameter ($A\exp(-Br-Cr^2)$).

 \vspace{0.15in} \Large
 \item  $\backslash$Cwill\{\} : \\ 
    \large
    Williams/Aziz-Chen parameter ($A\exp(-Br-Cr^2)$).

 \vspace{0.15in} \Large
 \item  $\backslash$rm\_swit\{\} : \\ 
    \large
    Williams/Aziz-Chen parameter controlling Vdw switching 
   ($f(r,r_m) = $).

\end{enumerate}

%-------------------------------------------------------------------
\clearpage
\huge
\item[] \underline{\bf \~{}bond\_parm}
\begin{enumerate}

 \vspace{0.15in} \Large
 \item  $\backslash$atom1\{\} : \\
    \large
    Atom type 1 (character data).
   
 \vspace{0.15in} \Large
 \item  $\backslash$atom2\{\} : \\ 
    \large
    Atom type 2 (character data).

 \vspace{0.15in} \Large
 \item  $\backslash$pot\_type\{{\bf harmonic},power-series,morse,null\} : \\
    \large
     Potential type of bond.

 \vspace{0.15in} \Large
 \item  $\backslash$fk\{\} : \\
    \large
     Force constant of bond in K/Angstro${\rm m}^2$.

 \vspace{0.15in} \Large
 \item  $\backslash$eq\{\} : \\
    \large
    Equilibrium bond length in Angstrom.

 \vspace{0.15in} \Large
 \item  $\backslash$eq\_res\{\} : \\
    \large
     Respa Equilibrium bond length in Angstrom.

 \vspace{0.15in} \Large
 \item  $\backslash$alpha\{\} : \\
    \large
    Morse parameter alpha in inverse Angstrom \\
    ($\phi(r) = d_0[\exp(-\alpha(r-r_0))-1]^2$).

 \vspace{0.15in} \Large
 \item  $\backslash$d0\{\} : \\
    \large
    Morse d0 in Kelvin
    ($\phi(r) = d_0[\exp(-\alpha(r-r_0))-1]^2$).

\end{enumerate}

%-------------------------------------------------------------------
\clearpage
\huge
\item[] \underline{\bf \~{}pseudo\_parm}
\begin{enumerate}

 \vspace{0.15in} \Large
 \item  $\backslash$atom1\{\} : \\
    \large
    Atom type in character data.

 \vspace{0.15in} \Large
 \item  $\backslash$vps\_typ\{local,kb,vdb,null\} : \\
    \large
    Pseudopotential type: local, Kleinman-Bylander, Vanderbilt or null.

 \vspace{0.15in} \Large
 \item  $\backslash$vps\_file\{\} : \\
    \large
    File containing the numerical generated pseudopotentials.

 \vspace{0.15in} \Large
 \item  $\backslash$n\_ang\{\} : \\
    \large
    Number of angular momentum projection operators required.

 \vspace{0.15in} \Large
 \item  $\backslash$loc\_opt\{\} : \\
    \large
    Which angular momentum pseudopotential is taken to be the local.

\end{enumerate}

%------------------------------------------------------------------
  \end{itemize}
%==================================================================

%==================================================================
%==================================================================
\clearpage
\begin{center}
\huge
{\bf Coordinate Input Files: } 
\end{center}
\large

At present the code will build topologies for the user. However,
it will not build coordinates. Therefore, the user must provide
an initial input coordinate file( see $\backslash$restart\_type\{initial\}
and $\backslash$in\_restart\_file\{sim\_restart.in\})
The first line of the restart file must contain the number of atoms,
followed by a one, followed by the number of path integral beads 
(usually 1). The code will grow beads and only P=1 need be provided
initially. The atoms positions then follow in the order specified by
the set file and the topology files. Finally, at the bottom of
the code the 3x3 simulation cell matrix must be specified.
The unit is Angstrom.

Example: \\ 
The molecular set file specifies 25 water molecules as molecule
type 1 and 3 bromine atom as molecule type 2. The water parameter file is 
written as OHH, i.e. oxygen is atom 1, hydrogen1 is atom 2, hydrogen2 is 
atom 3. Therefore, the input file should have 25 OHH coordinates followed
3 bromines. If the box is a square 25 angstrom on edge 
then the last three lines of the file should be \\
\hspace*{1.5in}  25  0  0  \\
\hspace*{1.5in}  0  25  0  \\
\hspace*{1.5in}  0  0   25 \\

%==================================================================
%==================================================================
\clearpage
\begin{center}
\huge
{\bf Parallel Decomposition \\ Keyword Dictionary: } 
\end{center}
\large

The following commands maybe specified in the parallel decomposition file:
\begin{enumerate}
\LARGE
\item {\bf \~{ }charm\_conf\_gen\_def[ ]}
\item {\bf \~{ }charm\_conf\_rho\_def[ ]}
\item {\bf \~{ }charm\_conf\_state\_def[ ]}
\item {\bf \~{ }charm\_conf\_PC\_def[ ]}
\item {\bf \~{ }charm\_conf\_NL\_def[ ]}
\item {\bf \~{ }charm\_conf\_map\_def[ ]}
\end{enumerate}
If no commands are set, the charm++ driver will decide values based
on system size and processor number and architecture. Below, the 
name nstate refers to the number of KS states and the name nproc
refers to the number of processors to be used.

%==================================================================
\begin{itemize}
%-------------------------------------------------------------------
\clearpage
\huge
\item []{\bf \~{ }charm\_conf\_gen\_def}
\begin{enumerate}
  \vspace{0.15in} \Large
  \item  $\backslash$useCommlib\{on:on,off \} : \\   \large
  Turn on/off Charm communication library use. Default on
  \vspace{0.15in} \Large
  \item  $\backslash$useCommlibMulticast\{ \} : \\   \large
  \vspace{0.15in} \Large
  Turn on/off Charm multicast communication library use. Default on
  \item  $\backslash$atmOutput\{on:on,off \} : \\   \large
  Turn on/off atom output. Default on
\end{enumerate}

%-------------------------------------------------------------------
\clearpage
\huge
\item[] {\bf \~{ }charm\_conf\_rho\_def}
\begin{enumerate}
  \vspace{0.15in} \Large
  \item $\backslash$gExpandFactRho\{1 \} : \\    \large
  Expands the g-space parallelization of the electron density beyond
  the number of planes by the specified non-integer multiplicative factor. 
  \vspace{0.15in} \Large
  \item $\backslash$rhoGHelpers\{1 \} : \\    \large
  Expands the g-space parallelization of the local pseudopotential
  evaluation by the integer increment.
  \vspace{0.15in} \Large
  \item $\backslash$rhoRsubplanes\{1 \} : \\    \large
  Expands the r-space parallelization of the exchange correlation and 
  local pseudopotential energies by the integer increment. 
  The latter only matters if EES is on.
  \vspace{0.15in} \Large
  \item $\backslash$rhoLineOrder\{skip:skip,none,random \} : \\    \large
  Static Load balance scheme for density g-space parallelization.
  \vspace{0.15in} \Large
  \item $\backslash$nchareHartAtmT\{1 : $1 \leq$ $\leq$ natmTyp\} : \\ \large
  Decomposition for the local pseudopotentil energy for use with the 
  EES method.
  \vspace{0.15in} \Large
  \item $\backslash$rhoSubPlaneBalance\{off:on/off \} : \\    \large
  When rhoRsubplanes$>1$, balance the communication.
  \vspace{0.15in} \Large
  \item $\backslash$rhoGToRhoRMsgCombine\{off:on/off \} : \\    \large
  When rhoRsubplanes$>1$, combine messages.
\end{enumerate}

%-------------------------------------------------------------------
\clearpage
\huge
\item[] {\bf \~{ }charm\_conf\_state\_def}
\begin{enumerate}
  \vspace{0.15in} \Large
  \item $\backslash$dataPath\{./STATES \} : \\    \large
  The directory in which the electronic state input files are
  stored. If the ``gen\_wave'' option is selected, the dataPath
  command is ignored.
  \vspace{0.15in} \Large
  \item $\backslash$dataPathOut\{./STATES\_OUT \} : \\    \large
  The directory in which the electronic state output files are stored.
  \vspace{0.15in} \Large
  \item $\backslash$gExpandFact\{1.0 \} : \\    \large
  Expands the g-space parallelization of the electronic states beyond the number of planes by
  the specified non-integer multiplicative factor. If the torus map option is
  employed, the product must be a power of 2, for example 
 (gexpandfact=2)x(nplane\_state=4)=8.
  \vspace{0.15in} \Large
  \item $\backslash$stateOutput\{on: on/off \} : \\    \large
  Turn the electronic state input on and off.
\end{enumerate}

%-------------------------------------------------------------------
\clearpage
\huge
\item[] {\bf \~{ }charm\_conf\_PC\_def}
\begin{enumerate}
  \vspace{0.15in} \Large
  \item $\backslash$sGrainSize\{ nstate\} : \\    \large
  Integer smatrix decomposition parameter. The default value is number of
  electron states, e.g. the matrix is not decomposed.
  \vspace{0.15in} \Large
  \item $\backslash$orthoGrainSize\{ nstate\} : \\    \large
  Integer lowdix decomposition parameter. The default value is number of
  electron states, e.g. the matrix is not decomposed. This parameter
  must be $\leq$ sGrainSize and ${\rm mod}(sGrainSize,oGrainSize)=0$.
  \vspace{0.15in} \Large
  \item $\backslash$numChunks\{ 1 \} : \\    \large
  Increase the g-space decomposition by the integer factor.
  \vspace{0.15in} \Large
  \item $\backslash$phantomSym\{ on:on/off\} : \\    \large
  Add some extra parallelization for Smatrix computations.
  \vspace{0.15in} \Large
  \item $\backslash$useOrthoHelpers\{ off:on/off\} : \\    \large
  Add some extra parallelization for Lowdin orthogonalization.
  Use only if (nstates/orthograinsize)$^2<$ nproc.
  \vspace{0.15in} \Large
  \item $\backslash$invsqr\_tolerance\{1e-15 \} : \\    \large
  Tolerence on the iterative matrix square root method.
  \vspace{0.15in} \Large
  \item $\backslash$invsqr\_max\_iter\{10 \} : \\    \large
  Maximum number of iteration for the matrix square root method.
\end{enumerate}

%-------------------------------------------------------------------
\clearpage
\huge
\item[] {\bf \~{ }charm\_conf\_NL\_def}
\begin{enumerate}
  \vspace{0.15in} \Large
  \item $\backslash$numSfGrps\{ 1\} : \\    \large
  Decompose the $N^3$ non-local computation into the specified number of groups.
  \vspace{0.15in} \Large
  \item $\backslash$numSfDups\{ \} : \\    \large
  Duplicate the $N^2$ evalulation of the structure factor for the
  $N^3$ non-local computation the specified number of times.
  \vspace{0.15in} \Large
  \item $\backslash$launchNLeesFromRho\{rs:rs,rhor,rhog \} : \\  \large
  Specify the launch point for the Non-local EES computation.
\end{enumerate}

%-------------------------------------------------------------------
\clearpage
\huge
\item []{\bf \~{ }charm\_conf\_map\_def}
\begin{enumerate}
  \vspace{0.15in} \Large
  \item $\backslash$Gstates\_per\_pe\{nstates \} : \\    \large
  If torusMap is off, specify the g-space state decomposition.
  \vspace{0.15in} \Large
  \item $\backslash$Rstates\_per\_pe\{nstates \} : \\    \large
  If torusMap is off, specify the r-space state decomposition.
  \vspace{0.15in} \Large
  \item $\backslash$torusMap\{on: on/off\} : \\    \large
  If your machine is a torus, this mapping scheme gives optimal performance.
  \vspace{0.15in} \Large
  \item $\backslash$loadMapFiles\{off:on/off \} : \\    \large
  Load pregenerated map files.
  \vspace{0.15in} \Large
  \item $\backslash$dumpMapFiles\{off:on/off \} : \\    \large
  Create map files.
  \vspace{0.15in} \Large
  \item $\backslash$fakeTorus\{off:on/off \} : \\    \large
  If your machine is a NOT torus, you can still use torus mapping.
  \vspace{0.15in} \Large
  \item $\backslash$torusDimX\{ nproc\} : \\    \large
  Fake Torus dimension along x in processors/node
  \vspace{0.15in} \Large
  \item $\backslash$torusDimY\{1 \} : \\    \large
  Fake Torus dimension along y in processors/node
  \vspace{0.15in} \Large
  \item $\backslash$torusDimZ\{1 \} : \\    \large
  Fake Torus dimension along z in processors/node
  \vspace{0.15in} \Large
  \item $\backslash$torusDimNX\{1 \} : \\    \large
  Fake Torus dimension along x in nodes
  \vspace{0.15in} \Large
  \item $\backslash$torusDimNY\{1 \} : \\    \large
  Fake Torus dimension along y in nodes
  \vspace{0.15in} \Large
  \item $\backslash$torusDimNZ\{ 1\} : \\    \large
  Fake Torus dimension along z in nodes
\end{enumerate}

%-------------------------------------------------------------------
\end{itemize}
%==================================================================


%==================================================================
%-----------------------------------------------------------------
\end{document}
%==================================================================


