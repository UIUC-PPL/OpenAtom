%% instruction for Martins-Troullier type pseudopotential
%
%\documentclass[letterpaper,12pt]{article}
%
%\usepackage{helvet}
%\usepackage{courier}
%
%
%
%\textwidth = 6.5 in
%\textheight = 9 in
%\oddsidemargin = 0.0 in
%\evensidemargin = 0.0 in
%\topmargin = 0.0 in
%\headheight = 0.0 in
%\headsep = 0.0 in
%\parskip = 0.2in
%\parindent = 0.0in
%
%
%\title{\fontfamily{phv}\selectfont{Converting pseudopotentials for OpenAtom}}
%%\author{\fontfamily{phv}\selectfont{Minjung Kim}}
%\date{}
%
%\begin{document}
%\maketitle

\section{Converting pseudopotentials for OpenAtom}


At this moment (May 2015),
\begin{itemize}
\item OpenAtom only supports norm-conserving pseudopotentials.
\item The converter works for CPMD pseudopotentials. You may want to download CPMD pseudopotential library (visit \verb+http://cpmd.org/download+.)
\end{itemize}
--- Minjung Kim

%%%%%%%%%%%%%%%%% MARTINS-TROULLIER 
\subsection{Martins-Troullier type pseudopotentials}


\noindent
\underline{ {\bf Source codes }(provided): }
	\begin{itemize}
	\item make{\_}mt{\_}pot.f
	\item piny{\_}make{\_}grid{\_}mt.f
	\end{itemize}
	
\vskip 12pt
\noindent
\underline{ {\bf Input files:} }
	\begin{itemize}
	\item potential file (e.g., Zn{\_}mt{\_}pot)
	\item wavefunction file (e.g., Zn{\_}mt{\_}wfn)
	\item PI{\_}MD.MAKE : This file contains input information needed to run piny{\_}make{\_}grid{\_}mt.f
	\end{itemize}

\vskip 12pt
\noindent
\underline{ {\bf Instruction:} }

\begin{enumerate}
\item Open CPMD pseudopotential file you wish to convert (e.g., Zn\_MT\_PBE\_SEMI.psp)
\item  Copy potential and wavefunction data from CPMD pseudopotential file (right after \verb+&POTENTIAL+ and \verb+&WAVEFUNCTION+) to potential (Zn{\_}mt{\_}pot) and wavefunction file (Zn\_mt\_wfn)
\item Edit make{\_}mt{\_}pot.f

        \begin{itemize}
	\item Enter the names of your potential and wavefunction files into the appropriate ``open" statement (line 3 and 4)
	\item Enter the names of your angular momentum channel files  you wish to create into the ``open" statement. 
	For example, if you have 3 angular momentum channels ($s, p, d$), then Zn{\_}l0, Zn{\_}l1, Zn{\_}l2
         \end{itemize}

\item Compile make{\_}mt{\_}pot.f and run. It will create Zn{\_}l0, Zn{\_}l1, Zn{\_}l2 files.

\item Edit PI{\_}MD.MAKE\\ The following informations are required:

	\begin{itemize}
	\item npts rmax lmax
	\item ZV
	\item alpha1 C1
	\item alpha2 C2
	\item l=0 file name 
	\item l=1 file name 
	\item l=2 file name 
	\item...
	\item l=lmax file name
	\end{itemize}
	
	npts: number of points on the ouput radial grid\\
	rmax: the largest r value you on the radial grid\\
	lmax: the highest angular momentum channel\\
	ZV: the valence charge\\
	alpha1, alpha2, C1, C2: parameters that determine the long-range part of the pseudopotential, assumed to be:\\
	
	-C1*erf(alpha1*r)/r - C2*erf(alpha2*r)/r\\
	
	We've been using alpha1=1.0 C1=1.0 and C2=0.0 so that the long-range part is -erf(r)/r

\item Compile piny{\_}make{\_}grid{\_}mt.f and run. It will create OpenAtom pseudopotential file.
\end{enumerate}



%%%%%%%%%%%%%%%%% Goedecker-Hutter type 
\subsection{Goedecker-Hutter type pseudopotential}

\noindent
\underline{ {\bf Source code }(provided): }
	\begin{itemize}
	\item make\_goedecker.c
	\end{itemize}

\vskip 12pt
\noindent
\underline{ {\bf Input file:} }
	\begin{itemize}
	\item input file (e.g., Cl\_SG\_BLYP\_INPUT -- see the example in the directory)
	\end{itemize}
	
	
\vskip 12pt
\noindent
\underline{ {\bf Instruction:} }

\begin{enumerate}
\item Create an input file which includes the following information:

First line -comments\\
Zion\\
ZV (valence electrons)\\
LMAX\\
rloc\\
number of C, C1, C2, C3, C4\\
Number of s channels\\
rs\\
h0(1,1)\\
h0(2,2)\\
Number of p channels\\
rp\\
h1(1,1)\\
Number of d channels\\
Number of f channels\\

You can get these numbers from CPMD PP file. Or look up PRB \textbf{58}, 3641 (1998) paper. All coefficients are tabulated.

\item Run \verb+make_goedecker.c+. You are prompted for the name of the input file, and it will create \verb+PSEUDO.OUT+.


\end{enumerate}




%\end{document}
