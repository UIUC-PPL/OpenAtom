%\documentclass[12pt,letterpaper,english]{article}
%\documentclass[prb,nobibnotes,floatfix,superscriptaddress,onecolumn,showpacs]{revtex4-1}
\documentclass[12pt,draft]{article}
\usepackage[T1]{fontenc} % Standard package for selecting font encodingsamely titles, headers and contents
%\usepackage{txfonts} % makes spacing between characters space correctly
\usepackage[top=1in, bottom=1in, left=1in, right=1in] {geometry} % Margins

%\usepackage{paralist}
\usepackage{graphicx}
\usepackage{xspace}
%\usepackage{longtable}
%\usepackage{subfigure}
%\usepackage{comment}
\usepackage{color}
%\usepackage{fancyhdr}% header footer placement
%\usepackage{ctable}
%\usepackage{booktabs}
%\usepackage{cite}
%\usepackage{wrapfig}
%\usepackage{subfigure}
%\usepackage{hyperref}
%\usepackage{multirow}
\usepackage{braket}
\usepackage{multirow}
\usepackage{amsmath}
\usepackage{enumerate}


\newcommand{\openatom}{\textsc{OpenAtom}}
\newcommand{\namd}{\textsc{NaMD}}
\newcommand{\cpaimd}{CPAIMD}
\newcommand{\charmpp}{\textsc{Charm}{\tt++}}
\newcommand{\rts}{RTS}
\newcommand{\mom}{MOPM}
\newcommand{\changa}{ChaNGa}
\newcommand{\petaflops}{peta{\sc flops}}
\newcommand{\Petaflops}{Peta{\sc flops}}
\newcommand{\megaflops}{mega{\sc flops}}
\newcommand{\Megaflops}{Mega{\sc flops}}
\newcommand{\teraflops}{tera{\sc flops}}
\newcommand{\Teraflops}{Tera{\sc flops}}
\newcommand{\CC}{C\kern -0.0em\raise 0.5ex\hbox{++}\xspace}
\newcommand{\orch}{Charisma\xspace}
\newcommand{\ampi}{AMPI\xspace}
\newcommand{\ppl}{PPL1\xspace}
\newcommand{\msa}{MSA\xspace }
\newcommand{\locus}{locus\xspace }
\newcommand{\loci}{loci\xspace }


\newcommand{\lociset}{LociSet\xspace }
\newcommand{\lociSet}{LociSet\xspace }
\newcommand{\indexset}{LociSet\xspace }
\newcommand{\indexSet}{LociSet\xspace }

\newcommand{\locisets}{LociSets\xspace }
\newcommand{\lociSets}{LociSets\xspace }
\newcommand{\indexsets}{LociSets\xspace }
\newcommand{\indexSets}{LociSets\xspace }
\definecolor{MyBlue}{rgb}{0.1,0.3,0.75}

\newcommand{\note}[1]{{\bf \large \color{red} #1}}
\newcommand{\red}{\color{red}}
\newcommand{\blue}{\color{blue}}
\newcommand{\minjung}{\color{gray}}
\newcommand{\black}{\color{black}}

% \newcommand{\comment}[1]{\textcolor{MyBlue}{\emph{#1}}}
% \newcommand{\red}[1]{\textcolor{red}{\emph{#1}}}
% %\newcommand{\comment}[1]{}

% \newcommand{\mycite}[1]{}

\newcommand{\jade}{\Jade}
\newcommand{\us}{\,$\mu$s\xspace }
\def\eg{{\it e.g.\ }}

%-------------------------------------------------------------------------


\begin{document}

\begin{center}
{\bf \Large Instruction Manual for \openatom~GW software}
\end{center}

\vspace{0.5in}




\noindent This document explains how to run 2-atom Si system with step-by-step instructions.

\noindent Example files with full data sets for small 2-atom unit cells of Si and GaAs are available via \openatom\ git repository: 
\\
\verb+git clone https://charm.cs.illinois.edu/gerrit/datasets/gwbse/Si2+\\
and\\
\verb+git clone https://charm.cs.illinois.edu/gerrit/datasets/gwbse/GaAs2+


\section{DFT calculations for states}
We currently support Quantum Espresso to generate ground state wavefunctions and energies.
In the example directory (datasets/Si2/QuantumEspresso), input files for `scf' and `bands' calculations are located. 
Once `bands' calculations are done, use converter to generate states that \openatom\ can read.
To prevent the Coulomb operator from diverging when $k=q=0$, you must generate wavefunctions and energies for occupied states with shifted k grids. A small shift is recommended (e.g., $(0,0,0.001)$ in our example). \\

\noindent\openatom~GW is designed for large systems which do not require dense k grid. No symmetries are considered in \openatom~GW. If you wish to run a small system with many k points, we recommend to use other GW softwares such as BerkeleyGW~\cite{BGW}. %\note{need citation}.  
For educational purposes, we present an example of 2-atom Si with 8 k points here. \\


\noindent Follow below steps to generate states files. Our example contains 8 k points and 52 states.
\begin{enumerate}

\item Run `scf' calculations (e.g., pw.x < in.scf)

\item Run `bands' calculations with desired number of empty states (e.g., pw.x < in.bands)

\item Creates \verb+STATES+ directory. Use \verb+makedir.sh+. \\
For this example, type \verb+./makedir.sh 8 1+\ on the command line

\item Run converter (pw2openatom.x < in.pw2openatom) with \verb+shift_flag=.false.+

\item Run `bands' calculations with shifted k grid (e.g. pw.x < in.bandsq). Only occupied states are needed.

\item Run converter  (pw2openatom.x < in.pw2openatom) with \verb+shift_flag=.true.+
 
\end{enumerate}

%\note{Does the converter create rho.dat?}
%yes it creates rho.dat if gpp\_flag=.true. but since we have not implemented GPP yet, I did not describe it in the documentation. \\

\noindent Description of the converter input options is available in \verb+README+ file in converter directory. 

\section{Input files for \openatom\ GW}
\noindent There are 4 input files that users need to create and specify values. All those files are available in the example directory.

\begin{enumerate}
\item \verb+klist.dat+ : specifying k lists and their weight as well as shifting vector
\item \verb+lattice.dat+ : specifying lattice vectors
\item \verb+band_list.dat+ : specifying which $\Sigma$ elements the user wants to calculate
\item simulation keywords file (\verb+config+ in this example) : the simulation keywords file controls all the options used in GW calculations. This file has any name.  
\end{enumerate}

\subsection{klist.dat}
\begin{verbatim}
8 : number of k-point
0.000000000  0.000000000  0.000000000   1.0    : k-point and weight starts 
0.000000000  0.000000000  0.500000000   1.0
0.000000000  0.500000000  0.000000000   1.0
0.000000000  0.500000000  0.500000000   1.0
0.500000000  0.000000000  0.000000000   1.0
0.500000000  0.000000000  0.500000000   1.0
0.500000000  0.500000000  0.000000000   1.0
0.500000000  0.500000000  0.500000000   1.0   : end
0.0  0.0  0.001 : specify the shifting vector
2    2    2     : density of k grid
\end{verbatim}

\subsection{lattice.dat}
\begin{verbatim}
10.2612 : lattice parameter in a.u.
 0.00   0.50   0.50 : start lattice vectors in unit of a 
 0.50   0.00   0.50
 0.50   0.50   0.00 : end
-1.00   1.00   1.00 : start reciprocal lattice vectors. in unit of 2pi/a
 1.00  -1.00   1.00 
 1.00   1.00  -1.00 : end
\end{verbatim}

\subsection{band\_list.dat}
This file sets which matrix element to calculate for $\Sigma$, <$\psi_{n_1}$|$\Sigma$|$\psi_{n_2}$>. Two numbers ($n_1$ and $n_2$) go into each line. This example will calculate diagonal $\Sigma$ for HOMO and LUMO. 
\begin{verbatim}
2 : number of matrix element to be evaluated
4  4 : first pair of n1 and n2
5  5 : second pair of n1 and n2
\end{verbatim}
%\note{what goes here?} 


\subsection{simulation keywords file (configuration file)}
The example of simulation keywords file is below. For the full keywords and key arguments list, please check with main documentation.
\begin{verbatim}

~gen_GW[
\gwbse_opt{on}       
\num_tot_state{52}       Total number of the states
\num_occ_state{4}       Number of the occupied states 
\num_unocc_state{48}    Number of the unoccupied states
\num_kpoint{8}    Number of k-point - should be consistent with klist.dat
\num_spin{1}    Number of spin
\statefile_binary_opt{off} 
]


~GW_epsilon[
\EcutFFT{12} FFT size 
\Ecuteps{10} Epsilon cutoff
]

~GW_sigma[
]

~GW_parallel[
\rows_per_chare{10}
\cols_per_chare{10}
\pipeline_stages{1}
]

\end{verbatim}

\section{Run \openatom\ GW}
Currently, the standard \openatom\ installation does not include GW executable. User must build \verb+gw_bse+  (\openatom\ GW does not require standard \openatom\ installation. This may change in the near future). If you have successfully compiled, inside of the \verb+/build+ directory, there are two execuatables: \verb+charmrun+ and \verb+gw_bse+. Copy these files to the dataset directory (\verb+/Si2+) and run with this command line:

\begin{verbatim}
 > ./charmrun +p1 ./gw_bse test_configuration
\end{verbatim}

\noindent If you want to get save the output into the file, you can do the standard output option:

\begin{verbatim}
 > ./charmrun +p1 ./gw_bse test_configuration > outfile
\end{verbatim}

\noindent You can change the number of processors by changing \verb+p1+ to \verb+pX+ where \verb+X+ is the number of processors.
The output of using 1 processor run is available in \verb+/output+ directory.


\begin{thebibliography}{111}
\bibitem{BGW} Deslippe, G. Samsonidze, D. A. Strubbe, M. Jain, M. L. Cohen and S. G. Louie, 	{\it Computer Physics Communications} {\bf 183}, 1269-1289, (2012).  DOI 10.1016/j.cpc.2011.12.006
\end{thebibliography}


\end{document}
